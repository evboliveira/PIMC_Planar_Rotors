%update
\documentclass[a4paper,11pt]{article}
\usepackage[top=2cm, left=1cm, right=1.5cm, bottom=2cm]{geometry}
\usepackage{indentfirst}
\usepackage{graphicx}% Include figure files
\usepackage{bbm}% bold math
\usepackage[pdftex,plainpages=false,colorlinks=true,citecolor=blue,linkcolor=blue,urlcolor=blue,filecolor=green,bookmarksopen=true]{hyperref}% add hypertext capabilities
\usepackage[utf8]{inputenc}
\usepackage[american]{babel}
\usepackage[T1]{fontenc}
\usepackage{amsthm}
\usepackage{amsmath}
\usepackage{amsfonts}
\usepackage{subfigure} 
\usepackage{enumerate}
\usepackage{physics}
\usepackage{comment}
\usepackage[usenames,dvipsnames]{xcolor}
\usepackage{multirow}
\usepackage{float}
\usepackage{lipsum}
\usepackage{tikz}
\usetikzlibrary{arrows, positioning, decorations.pathmorphing}
\usepackage{nicematrix}
\makeatletter
\renewcommand{\pm}{\mathbin{\mathpalette\@pm\relax}}
\newcommand{\@pm}[2]{\ooalign{%
		\raisebox{.4\height}{$#1+$}\cr
		\smash{\raisebox{-.3\height}{$#1-$}}\cr}}

\renewcommand{\mp}{\mathbin{\mathpalette\@mp\relax}}
\newcommand{\@mp}[2]{\ooalign{%
		\raisebox{.6\height}{$#1-$}\cr
		\smash{\raisebox{-.1\height}{$#1+$}}\cr}}
\makeatother

\usepackage{import}
\usepackage{xifthen}
\usepackage{pdfpages}
\usepackage{transparent}

\newcommand{\incfig}[1]{%
	\def\svgwidth{\columnwidth}
	\import{./figures/}{#1.pdf_tex}
}

%\usepackage{soul}
\newcommand{\mathcolorbox}[2]{\colorbox{#1}{$\displaystyle #2$}}
\definecolor{highlight}{rgb}{0.97,1,0.87}

\usepackage[most]{tcolorbox}
\tcbset{
	frame code={}
	center title,
	left=0pt,
	right=0pt,
	top=0pt,
	bottom=0pt,
	colback=highlight,
	colframe=black,
	width=\textwidth,
	enlarge left by=0mm,
	boxsep=0pt,
	arc=0pt,outer arc=0pt,
}

%----------New Commands------------% 

\newcommand\alert[1]{{\texttt{\textcolor{Red}{(#1)}}}}
\newcommand{\daga}{^\dagger}
%\newcommand{\oper}[1]{\boldsymbol{#1}}
\newcommand{\oper}[1]{\hat{#1}}
\newcommand{\1}{{\oper{I}}}
\renewcommand{\i}{{\mathbbm{i}}}
\renewcommand{\d}[1]{{\delta_{#1}}}
\renewcommand{\H}{\oper{\mathcal{H}}}
\newcommand{\s}{\oper{\sigma}}
\newcommand{\C}{\oper{C}}
\newcommand{\E}[2]{\oper{E}^{#1}_{#2}}
\renewcommand{\S}[2]{\oper{S}^{#1}_{#2}}
\newcommand{\M}{\oper{M}}
\renewcommand{\L}{\oper{L}}
\newcommand{\A}{\oper{A}}
\newcommand{\V}[1]{\oper{V}_{#1}}
\newcommand{\K}[1]{\oper{K}_{#1}}

\newcommand{\Nmath}{\mathbbm{N}}
\newcommand{\Zmath}{\mathbbm{Z}}
\newcommand{\Z}{\mathcal{Z}}
\renewcommand{\P}{\mathcal{P}}
\newcommand{\intdiv}{/\hspace{-0.07cm}/}


\newcommand{\m}[2]{%
	\ifthenelse{\isempty{#1}}%
	{\vb*{m}^{\hspace{-0.2mm}{#2}}}% if #2 is empty
	{m_{\hspace{-0.1mm}\scriptscriptstyle{#1}}^{\hspace{-0.2mm}\scriptscriptstyle{#2}}}% if #2 is not empty
}

\newcommand{\p}[2]{%
	\ifthenelse{\isempty{#1}}%
	{\boldsymbol{\varphi}^{\hspace{-0.2mm}\scriptscriptstyle{#2}}}% if #2 is empty
	{\varphi_{\hspace{-0.1mm}\scriptscriptstyle{#1}}^{\hspace{-0.2mm}\scriptscriptstyle{#2}}}% if #2 is not empty
}
\renewcommand{\sup}[1]{^{\scriptscriptstyle{#1}}}
\newcommand{\sub}[1]{_{\hspace{-0.01cm}\tiny{#1}}}
\newcommand{\mmax}{\overline{m}}
\newcommand{\nn}{\nonumber \\}
\renewcommand{\mod}[2]{~{#1}\,\,\text{mod}\,\,#2~}

\newcommand{\+}{\uparrow}
\renewcommand{\-}{\downarrow}
\newcommand{\0}{0}
\setcounter{tocdepth}{3}
\setcounter{secnumdepth}{3}


\title{PI QMC for Planar Rotors}
\author{Estevao de Oliveira}

\begin{document}
	
	\maketitle
	
	\label{Introduction}
	\label{sec:intro}
	In this manuscript the main goal is to describe the simulation of a quantum system of $N$ planar rotors using Quantum Monte Carl within the scope of the Path Integral method.
	
\section{Description of the physical system of $N$ quantum planar rotors -- $\ket{m}$ basis}
\label{sec:system_description}

The system of interest is constituted of $N$ identical linear rotors equally spaced with some distance $r$ in a linear chain, each one with its specific dipole momentum $\vec{\mu}$ and angular displacement $\varphi$ about the $x$-axis, as shown on the scheme of Fig. \ref{fig:planar_rotors_chain}. The kinetic energy of the $i$-th is 
\begin{align}
	\mathcal{K}_i = \frac{\mathcal{I}_i {\dot{\varphi}_i}^2}{2} = \frac{{L_i}^2}{2\mathcal{I}_i}  \, ,
\end{align}
with $L_i$ and $\mathcal{I}_i$ being its angular momentum and the moment of inertia, respectively. Since all the linear rotors are identical, we can define $\mathcal{I}_i \equiv \mathcal{I}$ for $i=1,2,\dots, N$. In addition, its potential energy, due to the dipole-dipole interaction is
\begin{align}
	\label{eq:potencial_energy_rotor_i}
	\mathcal{V}_i(\vec{r}_i)
	=-\frac{\mu_0}{4\pi} \sum_{j \neq i} \frac{3(\vec{\mu_i}\vdot \hat{r}_{ij})(\vec{\mu_j}\vdot \hat{r}_{ij})-\vec{\mu_i}\vdot\vec{\mu_j}}{\abs{\vec{r}_{ij}}^3} \, ,
\end{align}
where $\vec{r}_{ij} \equiv \vec{r}_{i} - \vec{r}_{j} = (i-j)r \hat{x}$. Besides, since $\hat{r}_{ij} \parallel \hat{x}$ up to a sign that cancels out on the terms of Eq. \eqref{eq:potencial_energy_rotor_i}, and the planar rotors have the same magnitude of the dipole moment, he will have that
\begin{align}
	\label{eq:potencial_energy_rotor_i_2}
	\mathcal{V}_i(\vec{r}_i)
	=-\frac{\mu_0}{4\pi} \frac{\mu^2}{r^3} \sum_{j \neq i} \frac{3(\hat{\mu}_i\vdot \hat{x})(\hat{\mu}_j\vdot \hat{x})-\hat{\mu}_i\vdot\vec{\mu}_j}{\abs{i-j}^3} \, ,
\end{align}
\begin{align}
	\mathcal{V}_i(\vec{r}_i)
	=&-\frac{\mu_0}{4\pi} \frac{\mu^2}{r^3} \sum_{j \neq i} \frac{3\cos{\varphi_i} \cos{\varphi_j}- \cos(\varphi_i-\varphi_j)}{\abs{i-j}^3}
	\nonumber \\ 
	=&\texttt{g}\sum_{j\neq i}^N \frac{\sin{\varphi_i} \sin{\varphi_j} -2\cos{\varphi_i} \cos{\varphi_j}}{\abs{i-j}^3} \, ,
\end{align}
where we defined $\texttt{g} \equiv \frac{\mu_0}{4\pi} \frac{\mu^2}{r^3}$. Since the kinetic energies is a quadratic function of the angular velocities $\dot{\varphi}_i$, and the potential energy only depends on the angular displacements ${\varphi}_i$, the classical Hamiltonian of the system is directly obtained
\begin{figure}[ht]
	\centering
	\incfig{planar_rotors_chain}
	\caption{Scheme showing a chain of $N$ planar rotors. The $\vec{\mu}_i$ and $\varphi_i$ corresponds to the dipole momentum (in blue) and the angular displacement (in red) about the $x$-axis, respectively, for the $i$-th planar rotor. The rotors are displaced in a co-planar arrangement \alert{cite Tobias paper}}
	\label{fig:planar_rotors_chain}
\end{figure}
\begin{align}
	H = \sum_{i=1}^{N} \frac{{L_i}^2}{2\mathcal{I}} + \sum_{i=1}^{N}\sum_{j>i}\texttt{g}\frac{\sin{\varphi_i} \sin{\varphi_j} -2\cos{\varphi_i} \cos{\varphi_j}}{\abs{i-j}^3} \, ,
\end{align}
where the $j>i$ condition on the potential energy term was put to avoid the double sum \cite{lemos2018analytical}. The Canonical Quantization is made by simply changing
\begin{align}
	\{\varphi_i,L_j\} = \d{ij} \longrightarrow \comm{\oper{\varphi}_i}{\oper{L}_j} = \i \hbar \1 \, ,
\end{align}
for $\1$ being the Identity matrix of the same dimension as $\oper{L}_j$, resulting on the following Hamiltonian	
\begin{align}
	\label{eq:Hamiltonian}
	\oper{H} = \sum_{i=1}^{N} \frac{1}{2\mathcal{I}} \oper{L}_i^2 + \sum_{i=1}^{N}\sum_{j>i}^{N}\texttt{g}\frac{\sin{\oper{\varphi}_i}\sin{\oper{\varphi}_j} - 2\cos{\oper{\varphi}_i}\cos{\oper{\varphi}_j}}{\abs{i-j}^3} \, .
\end{align}
It is known that, in the angular momentum basis we have
\begin{align}
	\oper{L}\ket{m} &= \hbar m \ket{m} \, , \\
	\oper{L}^2\ket{m} &= \hbar^2 m^2 \ket{m} \, ,
\end{align}
for $m \in \Zmath$. However, for the terms $\cos{\oper{\varphi}}$ and $\sin{\oper{\varphi}}$ it may be tricky to define the matrix elements in the $\{\ket{m}\}$ basis when regarding $\oper{\varphi}$ as the operator. Here, it is convenient to define
\begin{align}
	\cos{\oper{\varphi}} &= \frac{e^{\i \oper{\varphi}}+e^{-\i \oper{\varphi}}}{2} \, , \\
	\sin{\oper{\varphi}} &= \frac{e^{\i \oper{\varphi}}-e^{-\i \oper{\varphi}}}{2\i} \, ,
\end{align}
so that
\begin{align}
	\comm{\oper{L}}{e^{\pm\i \oper{\varphi}}} 
	=\sum_{n=0}^{\infty} \frac{(\pm \i)^n}{n!} \comm{\oper{L}}{\oper{\varphi}^n}
	=\sum_{n=0}^{\infty} \frac{(\pm \i)^n}{n!} ~ n (-\i) \oper{\varphi}^{n-1} 
	= (\pm \i)(- \i) \sum_{n=1}^{\infty} \frac{(\pm \i)^{n-1}}{(n-1)!} \oper{\varphi}^{n-1}
	= \pm \sum_{p=0}^{\infty} \frac{(\pm \i)^{p}}{p!} \oper{\varphi}^{p}
	= \pm e^{\pm\i \oper{\varphi}} \, .
\end{align}
It is also obvious to see that $\comm{\oper{\varphi}}{e^{\pm\i \oper{\varphi}}}=0$, and $e^{\pm\i \oper{\varphi}}\cdot e^{\mp\i \oper{\varphi}} = \1 \implies \comm{e^{+\i \oper{\varphi}}}{e^{-\i \oper{\varphi}}}=0$. Moreover, 
\begin{align}
	\comm{\oper{L}}{e^{\pm\i \oper{\varphi}}} \ket{m} =& \pm e^{\pm\i \oper{\varphi}} \ket{m} \nn
	\oper{L} e^{\pm\i \oper{\varphi}} \ket{m} -  e^{\pm\i \oper{\varphi}} \oper{L}\ket{m} =& \pm e^{\pm\i \oper{\varphi}} \ket{m} \nn
	\oper{L} e^{\pm\i \oper{\varphi}} \ket{m}	 - m e^{\pm\i \oper{\varphi}} \ket{m}  =& \pm e^{\pm\i \oper{\varphi}} \ket{m} 
	\nn
	\oper{L} \, e^{\pm\i \oper{\varphi}} \ket{m} =& (m \pm 1) \, e^{\pm\i \oper{\varphi}} \ket{m}  \, ,
\end{align}
implying that the $e^{\pm\i \oper{\varphi}} \ket{m}$ is an eigenstate of the $\L$ operator, with eigenvalue $m\pm 1$. Hence, the action of the $e^{\pm\i \oper{\varphi}}$ operators on the momentum state $\ket{m}$ can be given by
\begin{align}
	\label{eq:creat_annhi_ops}
	e^{\pm\i \oper{\varphi}} \ket{m} =& \ket{m \pm 1} \,.
\end{align}
Here, in order to simplify the notation we can define $e^{\pm\i \oper{\varphi}} \equiv \E{\pm}{}$, so that
\begin{align}
	\L \ket{m} =& m\ket{m} \, , \label{eq:L_action}\\ 
	\E{\pm}{} \ket{m} =& \ket{m \pm 1} \,, \label{eq:E_pm_action}\\
	\comm{\L}{\E{\pm}{}} =& \pm \E{\pm}{} \,,\label{eq:L,E_pm_comm}\\
	\E{+}{}\E{-}{} = \E{-}{}\E{+}{} = \1 & \implies\comm{\E{+}{}}{\E{-}{}} = 0 	\label{eq:Ep,Em_comm}\, . 
\end{align}

Then, the dipole-dipole interaction can be written as 
\begin{align}
	\label{eq:dipole-dipole_interaction}
	\V{ij} 
	=& \frac{\texttt{g}}{\abs{i-j}^3} \left[\frac{\E{+}{i}-\E{-}{i}}{2\i} \cdot \frac{\E{+}{j}-\E{-}{j}}{2\i} - 2\frac{\E{+}{i}+\E{-}{i}}{2} \cdot \frac{\E{+}{j}+\E{-}{j}}{2}\right] \nn
	=&- \frac{\texttt{g}}{4 \abs{i-j}^3} \left[\left(\E{+}{i}-\E{-}{i}\right)\left(\E{+}{j}-\E{-}{j}\right) + 2\left(\E{+}{i}+\E{-}{i}\right)\left(\E{+}{j}+\E{-}{j}\right)\right] \nn
	=&- \frac{\texttt{g}}{4 \abs{i-j}^3} \left[\E{+}{i}\E{+}{j}-\E{+}{i}\E{-}{j}-\E{-}{i}\E{+}{j}+\E{-}{i}\E{-}{j} + 2\E{+}{i}\E{+}{j}+2\E{+}{i}\E{-}{j}+2\E{-}{i}\E{+}{j}+2\E{-}{i}\E{-}{j}\right] \, ,
\end{align}
and the system Hamiltonian in the Eq. \eqref{eq:Hamiltonian} becomes \cite{eh_ek_2008,Lopez_Vazquez_2016}
\begin{align}
	\label{eq:Hamiltonian_final}
	\boxed{
		\mathcolorbox{highlight}{
			\H = \frac{\oper{H}}{B} = \sum_{i=1}^{N} \L_i^2 - \frac{g}{4} \sum_{i=1}^{N}\sum_{j>i}^N\frac{3\E{+}{i}\E{+}{j}+\E{+}{i}\E{-}{j}+\E{-}{i}\E{+}{j}+3\E{-}{i}\E{-}{j}}{\abs{i-j}^3} \, ,
	}}
\end{align}
where $g=\frac{\texttt{g}}{B}$ our dimensionless interaction strength parameter, for the rotational constant $B=\frac{\hbar^2}{2\mathcal{I}}$.
We can see that the operators from eq.\eqref{eq:Hamiltonian_final} assume the matrix form
\begin{equation}
	\label{eq:matrix_rep_L}
	\L =\sum_{m= -\infty}^{+\infty} m \ketbra{m}{m}
	=
	\begin{pNiceMatrix}
		\Ddots[shorten=0.3cm]\hspace*{-0.2cm}&&&&&&\\
		&+2&0 &  &  &0 &\\[0.2cm]
		&0 &+1&  &  &  &\\[0.2cm]
		&  &  &0 &  &  &\\[0.2cm]
		&  &  &  &-1&0 &\\[0.2cm]
		&0 &  &  &0 &-2&\\
		&  &  &  &  &  &\Ddots[shorten=0.3cm]\hspace*{-0.2cm}
		\CodeAfter \line{3-2}{6-2}
		\CodeAfter \line{3-2}{6-5}
		\CodeAfter \line{6-2}{6-5}			
		\CodeAfter \line{2-3}{2-6}			
		\CodeAfter \line{2-3}{5-6}						
		\CodeAfter \line{2-6}{5-6}
	\end{pNiceMatrix}
	%		\Block[c]{2-2}<\huge>{0}
	\, ,
\end{equation}
\begin{equation}
	\label{eq:matrix_rep_Ep}
	\E{+}{} =\sum_{m= -\infty}^{+\infty} \ketbra{m+1}{m}=	
	\begin{pNiceMatrix}
		\Ddots[shorten=0.3cm]\hspace*{-0.2cm}&&&&&&\\
		&0&1&0&&0&\\
		&&&1&&&\\
		&&&&1&0&\\
		&&&&&1&\\
		&0&&&&0&\\
		&&&&&&\Ddots[shorten=0.3cm]\hspace*{-0.2cm}
		\CodeAfter \line{2-2}{6-2}
		\CodeAfter \line{2-2}{6-6}
		\CodeAfter \line{6-2}{6-6}			
		\CodeAfter \line{2-4}{2-6}			
		\CodeAfter \line{2-4}{4-6}						
		\CodeAfter \line{2-6}{4-6}
	\end{pNiceMatrix}	 	
	\, ,
\end{equation}
\begin{equation}
	\label{eq:matrix_rep_Em}
	\E{-}{} =\sum_{m= -\infty}^{+\infty} \ketbra{m-1}{m}=
	\begin{pNiceMatrix}
		\Ddots[shorten=0.3cm]\hspace*{-0.2cm}&&&&&&\\
		&0&&&&0&\\
		&1&&&&&\\
		&0&1&&&&\\
		&&&1&&&\\
		&0&&0&1&0&\\
		&&&&&&\Ddots[shorten=0.3cm]\hspace*{-0.2cm}
		\CodeAfter \line{2-2}{2-6}
		\CodeAfter \line{2-2}{6-6}
		\CodeAfter \line{2-6}{6-6}			
		\CodeAfter \line{4-2}{6-2}			
		\CodeAfter \line{4-2}{6-4}						
		\CodeAfter \line{6-2}{6-4}
	\end{pNiceMatrix}
	\, ,
\end{equation}
given the angular momentum basis $\{\ket{m}\}$.

\subsection{Truncating the infinite discrete space generated by $\{\ket{m}\}$}
The angular momentum basis $\{\ket{m}\}$ form a infinite discrete orthonormal set, so in order to make computation analysis feasible it is necessary to truncate the space generated by $\{\ket{m}\}$ setting a maximum value of $m$, defined as $\mmax$. Then, the operators $\L$ and $\E{\pm}{}$ will become
\begin{equation}
	\L =\sum_{m= -\mmax}^{+\mmax} m \ketbra{m}{m}	 	
	=
	\begin{pNiceMatrix}
		+\mmax                    &   & \Block[c]{1-1}<\LARGE>{0} \\[0.2cm]
		& 0 &                           \\
		\Block[c]{1-1}<\LARGE>{0} &   & -\mmax
		\CodeAfter \line{1-1}{2-2}
		\CodeAfter \line{2-2}{3-3}
	\end{pNiceMatrix} \, ,
\end{equation}	
\begin{equation}
	\E{+}{} = {\color{red} \ketbra{-\mmax}{+\mmax}} +  \sum_{m= -\mmax}^{+\mmax-1} \ketbra{m+1}{m}	
	=\begin{pNiceMatrix}
		0&1&0&~~&0\\
		&&1&&\\
		&&&&0\\
		&&&&1\\
		{\color{red} 1}&&&&0
		\CodeAfter \line{1-1}{5-1}
		\CodeAfter \line{1-1}{5-5}
		\CodeAfter \line{5-1}{5-5}			
		\CodeAfter \line{1-3}{1-5}			
		\CodeAfter \line{1-3}{3-5}						
		\CodeAfter \line{1-5}{3-5}									
		\CodeAfter \line{2-3}{4-5}[shorten=0.3cm]
	\end{pNiceMatrix} 	
	\, ,
\end{equation}
\begin{equation}
	\E{-}{} = {\color{red} \ketbra{+\mmax}{-\mmax}} +  \sum_{m= -\mmax+1}^{+\mmax} \ketbra{m-1}{m}
	=\begin{pNiceMatrix}
		0&&&&{\color{red} 1}\\
		1&&&&\\
		0&1&&&\\
		&&&&\\
		0&&0&1&0
		\CodeAfter \line{1-1}{1-5}
		\CodeAfter \line{1-1}{5-5}
		\CodeAfter \line{1-5}{5-5}			
		\CodeAfter \line{3-1}{5-1}			
		\CodeAfter \line{3-1}{5-3}						
		\CodeAfter \line{5-1}{5-3}								
		\CodeAfter \line{3-2}{5-4}[shorten=0.3cm]
	\end{pNiceMatrix}	 	
	\, ,
\end{equation}
where the terms in red are added \textit{ad-hoc} so that the condition of Eq. \eqref{eq:Ep,Em_comm} holds.

\section{Canonical ensemble description of a Quantum Many-Body system}
	Recalling from statistical mechanics, a quantum system is in thermal equilibrium with a reservoir at a fixed temperature $T$ can be described by the canonical ensemble, with the canonical partition function can be written as
	\begin{align}
		\label{eq:partition_function}
		\Z = \Tr(e^{-\beta \H}) \, ,
	\end{align}
	for $\beta \equiv \frac{1}{\kappa_B T}$, for $\kappa_B$ the Boltzmann constant, and Hamiltonian $\H$ being the Hamiltonian quantum operator f the system. Choosing some convenient basis $\{\ket{\m{}{}}\}$ in which the quantum states stand for some physical property $\m{}{}$, Eq. \eqref{eq:partition_function} becomes
	\begin{align}
		\label{eq:partition_function_def}
		\Z = \sum_{\m{}{}} \mel**{\m{}{}}{e^{-\beta \H}}{\m{}{}} \, ,
	\end{align}
	Then, our goal will be to calculate the expectation value of quantities of interest as
	\begin{align}
		\label{eq:expectation_value}
		\expval{\oper{O}} = \frac{1}{\Z} \sum_{\m{}{}} \mel**{\m{}{}}{e^{-\beta \H} \oper{O}}{\m{}{}} \, ,
	\end{align}
	for some physical observable $\oper{O}$, and 
	\begin{align}
		\label{eq:def_basis_states}
		\ket{\m{}{}} = \bigotimes_{k=1}^N \ket{\m{k}{}} \, ,
	\end{align}
	with $\ket{\m{k}{}}$ representing the quantum state of the $k$-th particle in the system.
	
\section[PIMC method for a system of N quantum planar rotors -- ang. mom. basis]{PIMC method for a system of $N$ quantum planar rotors -- $\ket{m}$ basis}
	
	Here, we are going to represent both the Path Integral Ground State (PIGS) and the Path Integral for Finite Temperature (PIMC) since the derivation for the estimators is similar on both. For the PIGS formulation some \textbf{trial state vector} $\ket{\Psi_T}$ (not necessarily normalized and with the condition of being non-orthogonal to the true ground state of the system) evolves according to the propagation in imaginary time $\beta$ \cite{Yan_2017} (see \href{https://en.wikipedia.org/wiki/Wick_rotation}{Wick rotation})
	\begin{align}
		\ket{\Psi_{\beta}} =& e^{-\frac{\beta}{2} \H} \ket{\Psi_T} \, ,
	\end{align}
	where the ground state of the system can be obtained in the limit of $\beta \rightarrow \infty$,
	\begin{align}
		\label{eq:ground_state_PIGS}
		\ket{\Psi_{gs}} = \lim_{\beta \rightarrow \infty} e^{-\frac{\beta}{2} \H} \ket{\Psi_T} \,.
	\end{align}
	Then, the expectation value for some operator $\expval{\oper{O}}$ becomes
	\begin{align}
		\label{eq:expectation_PIGS}
		\expval{\oper{O}} =& \frac{\mel**{\Psi_{\beta}}{\oper{O}}{\Psi_{\beta}}}{\braket{\Psi_{\beta}}} \nonumber \\
		=& \frac{ \mel**{\Psi_T}{e^{-\frac{\beta}{2} \H} \oper{O}e^{- \frac{\beta}{2} \H }}{\Psi_T}}{\mel**{\Psi_T}{e^{-\beta \H}}{\Psi_T}}
		 \, ,
	\end{align}
	where now the (pseudo)-partition function can be defined
	\begin{align}
		\label{eq:pseudo_partition_function}
		\Z(\beta) = \mel**{\Psi_T}{e^{-\beta \H}}{\Psi_T} \,, 
	\end{align}
	which can be regarded as a particular case of Eq. \eqref{eq:partition_function_def} for the case here the sum on $\m{}{}$ is changed by the trial function $\ket{\Psi_T}$, with the \textit{ad-hoc} insertion of the Kronecker deltas $\delta{\Psi_T,\m{}{}}$.
	
	Now, taking Eq. \eqref{eq:partition_function_def} and dividing $\beta$ in small intervals	
	\begin{align}
		\label{eq:partition_function_PIMC}
		\Z(\beta) =& \Tr(e^{-\beta \H}) \nn
		=& \sum_{\m{}{1}} \mel**{\m{}{1}}{e^{-\beta \H}}{\m{}{1}} \nn
		=& \sum_{\m{}{1}} \mel**{\m{}{1}}{e^{-\frac{\beta}{L} \H}  e^{-\frac{\beta}{L} \H} \dots e^{-\frac{\beta}{L} \H}}{\m{}{1}} \nn
		=& \sum_{\m{}{1}} \sum_{\m{}{2}} \sum_{\m{}{3}} \dots \sum_{\m{}{L}}\mel**{\m{}{1}}{e^{-\tau \H}}{\m{}{2}}\mel**{\m{}{2}}{e^{-\tau \H}}{\m{}{3}} \dots \mel**{\m{}{L}}{e^{-\tau \H}}{\m{}{1}} \nn
		=& \sum_{\{\m{}{l}\}_{L}} \prod_{l=1}^{L} \mel**{\m{}{l}}{e^{-\tau \H}}{\m{}{l+1}}
		\nn
		=& \sum_{\{\m{}{l}\}_{L+1}} \prod_{l=1}^{L} \mel**{\m{}{l}}{e^{-\tau \H}}{\m{}{l+1}} \delta_{\m{}{L+1},\m{}{1}}
	\, ,
	\end{align} 
	\begin{align}
		\Z(\beta) 
		=& \sum_{\m{}{1}} \mel**{\m{}{1}}{e^{-\beta \H}}{\m{}{1}} \nn
		=& \sum_{\m{}{1}} \sum_{\m{}{2}} \sum_{\m{}{3}} \dots \sum_{\m{}{L}}\mel**{\m{}{1}}{e^{-\tau \H}}{\m{}{2}}\mel**{\m{}{2}}{e^{-\tau \H}}{\m{}{3}} \dots \mel**{\m{}{L}}{e^{-\tau \H}}{\m{}{1}} \nn
		\approx& \sum_{\{\m{}{l}\}_L} e^{-\tau(\m{}{1})^2} \mel**{\m{}{1}}{e^{-\tau \V{}}}{\m{}{2}} e^{-\tau(\m{}{2})^2}\mel**{\m{}{2}}{e^{-\tau \V{}}}{\m{}{3}} \dots e^{-\tau(\m{}{L})^2}\mel**{\m{}{L}}{ e^{-\tau \V{}}}{\m{}{1}} \nn
		\, ,
	\end{align}
	for $\tau \equiv \frac{\beta}{L}$, $L \in \Nmath$, and $\ket{\m{}{L+1}}= \ket{\m{}{1}}$ ($\ket{\m{}{L+1}} = \ket{\m{}{1}} = \ket{\Psi_T}$ for PIGS). 
	\subsection{Trotter Expansion $\H = \K{} + \V{}$}
	
	The Hamiltonian can be represented as
	\begin{align}
		\H 
		=& \sum_{i=1}^{N} \K{i} + \sum_{i=1}^{N-1} \V{i,i+1} \, ,
	\end{align}
	and since $\comm{\K{i}}{\V{i,i+1}} \neq 0$ (see appendix \ref{sec:commutators_calculations}) we need to apply the \textbf{Trotter expansion} to exponential function in order to have
	\begin{align}
		e^{-\tau \H} = e^{-\tau (\K{} + \V{})} \approx e^{-\tau \K{}} e^{-\tau \V{}} \, ,
	\end{align}
	so that Eq. \eqref{eq:partition_function_PIMC} becomes
	\begin{align}
		\label{eq:partition_function_PIMC_trotter_K+V}
		\Z(\beta) 
		\approx& \sum_{\{\m{}{l}\}_{L+1}} \prod_{l=1}^{L} \mel**{\m{}{l}}{e^{-\tau \K{}}e^{-\tau \V{}}}{\m{}{l+1}} \delta_{\m{}{L+1},\m{}{1}} \nn
		=& \sum_{\{\m{}{l}\}_{L+1}} \prod_{l=1}^{L} e^{-\tau(\m{}{l})^2} \mel{\m{}{l}}{ e^{ -\tau \V{}}}{\m{}{l+1}}  \delta_{\m{}{L+1},\m{}{1}}
		\, ,
	\end{align}
	where we will define $ \rho_{K}^l \equiv e^{-\tau(\m{}{l})^2} = e^{-\tau\sum_{n=1}^{N}(\m{n}{l})^2}$. Now, when analyzing the potential term, we notice that $\comm{\V{i,i+1}}{\V{i+1,i+2}} = 0$ (see appendix \ref{sec:commutators_calculations}), so it is convenient to split the sum into only two terms as
	\begin{equation}
		\V{} = \V{odd} + \V{even} 
		\text{ , for }
		\left\{
		\begin{aligned}
			&\V{odd} \equiv \V{12} + \V{34} + \dots =  \sum_{n \text{ odd}}^{N-1} \V{n,n+1} \, ,\\
			&\V{even} \equiv \V{23} + \V{45} + \dots =\sum_{n \text{ even}}^{N-1} \V{n,n+1}  \, .
		\end{aligned}
		\right.
	\end{equation}
	Therefore, $\comm{\V{odd}}{\V{even}} = 0 \implies e^{ -\tau \V{}} = e^{ -\tau \V{odd}} e^{ -\tau \V{even}}$, and after inserting a resolution of identity Eq. \eqref{eq:partition_function_PIMC_trotter_K+V} becomes
	\begin{align}
		\Z
		=& \sum_{\{\m{}{l}\}_{L}} \prod_{l=1}^{L} \rho_{K}^l \mel{\m{}{l}}{ e^{ -\tau \V{odd}} {\color{red} \sum_{\m{}{\alpha}} \ketbra{\m{}{\alpha}}} e^{ -\tau \V{even}}}{\m{}{l+1}}
		\nn
		=& \sum_{\{\m{}{l}\}_{L}} \prod_{l=1}^{L} \rho_{K}^l \sum_{\m{}{\alpha}} \mel**{\m{}{l}}{ e^{ -\tau \V{odd}}}{\m{}{\alpha}} \mel**{\m{}{\alpha}}{e^{ -\tau \V{even}}}{\m{}{l+1}}
		\nn
		=& \sum_{\{\m{}{l}\}_{L}}
		\rho_{K}^l \left(\sum_{\m{}{\alpha\sub{1}}}   \mel{\m{}{1}}{ e^{ -\tau \V{odd}}}{\m{}{\alpha\sub{1}}} \mel{\m{}{\alpha\sub{1}}}{e^{ -\tau \V{even}}}{\m{}{2}}\right) \times \nn
		& \hspace{0.55cm} \times
		\rho_{K}^2 \left(\sum_{\m{}{\alpha\sub{2}}}  \mel{\m{}{2}}{ e^{ -\tau \V{odd}}}{\m{}{\alpha\sub{2}}} \mel{\m{}{\alpha\sub{2}}}{e^{ -\tau \V{even}}}{\m{}{3}}\right) \times \dots \nn		
		& \hspace{0.08cm} \dots \times
		\rho_{K}^L \left(\sum_{\m{}{\alpha\sub{L}}} \mel{\m{}{L}}{ e^{ -\tau \V{odd}}}{\m{}{\alpha\sub{L}}} \mel{\m{}{\alpha\sub{L}}}{e^{ -\tau \V{even}}}{\m{}{L+1}}\right)
		\nn
		%
		=& \sum_{\m{}{1}} \sum_{\m{}{2}} \sum_{\m{}{3}}\dots  \sum_{\m{}{L}} ~~
		\sum_{\m{}{\alpha\sub{1}}} \sum_{\m{}{\alpha\sub{2}}} \sum_{\m{}{\alpha\sub{3}}} \dots \sum_{\m{}{\alpha\sub{L}}} 
		\prod_{l=1}^{L} \rho_{K}^l \mel**{\m{}{l}}{e^{ -\tau \V{odd}}}{\m{}{\alpha\sub{l}}}
		\mel**{\m{}{\alpha\sub{l}}}{e^{ -\tau \V{even}}}{\m{}{l+1}}
		\, ,
	\end{align} 
	Then, reorganizing the sums and relabelling the terms as $l \rightarrow 2p-1$, $\alpha\sub{l} \rightarrow 2p$, and $l+1 \rightarrow 2p+1$ we have
	\begin{align}
		\label{eq:partition_function_trotter_K+V}
		\Z(\beta)
		=& \sum_{\{\m{}{p}\}_{2L}}
		\prod_{p=1}^{L} \rho_{K}^l\mel**{\m{}{2p-1}}{e^{ -\tau \V{odd}}}{\m{}{2p}}
		\mel**{\m{}{2p}}{e^{ -\tau \V{even}}}{\m{}{2p+1}} \nn
		\Z(\beta)=& \sum_{\{\m{}{p}\}_{2L+1}}
		\prod_{p=1}^{2L}  \mel**{\m{}{p}}{\oper{\rho}(p)}{\m{}{p+1}}  \delta_{\m{}{2L+1},\m{}{1}}
		\, ,
	\end{align}
	where the operator $\oper{\rho}(p)$ is defined as
	\begin{equation}
		\label{eq:def_rho_tau_op_K+V}
		\oper{\rho}(p) =
		\left\{
		\begin{aligned}
			&\rho_{K}^p e^{ -\tau \V{odd}} & \text{, for $p$ odd,} \\
			&e^{ -\tau \V{even}} & \text{, for $p$ even.} 			
		\end{aligned}
		\right.
	\end{equation}
	Calculating each term of the matrix elements we have,
	\begin{align}
		\mel**{\m{}{p}}{ \rho_{K}^p e^{ -\tau \V{odd}}}{\m{}{p+1}}
		=& \rho_{K}^p\mel**{\m{}{p}}{e^{ -\tau \sum_{n \text{ odd}}^{N-1} \V{n,n+1}}}{\m{}{p+1}} \nn
		=& \rho_{K}^p\mel{\m{}{p}}{\prod_{n \text{ odd}}^{N-1} e^{ -\tau \V{n,n+1}} }{\m{}{p+1}}\nn
		=& \rho_{K}^p\prod_{n \text{ odd}}^{N-1} \mel**{\m{n}{p},\m{n+1}{p}}{e^{ -\tau \V{n,n+1}}}{\m{n}{p+1},\m{n+1}{p+1}} \, ,
	\end{align}
	where we have omitted all the delta functions regarding the other states to keep the notation simplified. Also, analogously
	\begin{align}
		\mel**{\m{}{p}}{e^{ -\tau \V{even}}}{\m{}{p+1}}
		=& \prod_{n \text{ even}}^{N-1}\mel{\m{n}{p},\m{n+1}{p}}{e^{ -\tau \V{n,n+1}}}{\m{n}{p+1},\m{n+1}{p+1}}  \, .
	\end{align}
	Then, the partition function becomes
	\begin{align}
		\label{eq:part_function_expanded_K+V}
		\boxed{
			\mathcolorbox{highlight}{
				\Z(\beta)
				= \sum_{\{\m{}{p}\}_{2L}}
				\prod_{p=1}^{2L} \prod_{n \in \mathcal{A}_p} \mel{\m{n}{p},\m{n+1}{p}}{}{\m{n}{p+1},\m{n+1}{p+1}}
				\, ,
		}}
	\end{align}
	for set $\mathcal{A}_p$ off all odd (even) numbers in the interval $\{1,N-1\}$ if $p$ is an odd (even) number, and where the double bar notation $||$ now makes implicit the matrix element being taken with respect to the $\oper{\rho}(p)$ as defined on Eq. \eqref{eq:def_rho_tau_op_K+V}. 
	
	As an example, we can write Eq. \eqref{eq:part_function_expanded} for $N=5$ planar rotor
	\begin{align}
		\label{eq:part_function_expanded_N=5_ex}
		\Z(\beta)
		=& \sum_{\{\m{}{p}\}_{2L}}
		\prod_{p=1}^{2L} \prod_{n \in\mathcal{A}_p } \mel{\m{n}{p},\m{n+1}{p}}{}{\m{n}{p+1},\m{n+1}{p+1}}
		\nn
		=& \sum_{\{\m{}{p}\}_{2L}}
		{\color{red} e^{-\tau\sum_{n=1}^{5}(\m{n}{1})^2}}
		\mel{\m{1}{1},\m{2}{1}}{}{\m{1}{2},\m{2}{2}}
		\mel{\m{3}{1},\m{4}{1}}{}{\m{3}{2},\m{4}{2}}
		\times 
		\mel{\m{2}{2},\m{3}{2}}{}{\m{2}{3},\m{3}{3}}
		\mel{\m{4}{2},\m{5}{2}}{}{\m{4}{3},\m{5}{3}} 
		\times \nn
		& \times
		{\color{red} e^{-\tau\sum_{n=1}^{5}(\m{n}{3})^2}}
		\mel{\m{1}{3},\m{2}{3}}{}{\m{1}{4},\m{2}{4}}
		\mel{\m{3}{3},\m{4}{3}}{}{\m{3}{4},\m{4}{4}}
		\times \dots \times
		\mel{\m{2}{2L},\m{3}{2L}}{}{\m{2}{1},\m{3}{1}}
		\mel{\m{4}{2L},\m{5}{2L}}{}{\m{4}{1},\m{5}{1}}
		\, ,
	\end{align}
	which can be graphically depicted as in the Fig. \ref{fig:PIMC_grid_N=5}.
	\begin{figure}[ht]
		\centering
		\incfig{PIMC_grid_N=5}
		\caption{Scheme showing the graphical representation of the grid defined on eq. \eqref{eq:part_function_expanded_N=5_ex} for a system of $N=5$ planar rotors, where the dots represents the particles, and the hollow rectangles, the two particle operators. The red colour stands for the kinetic energy of the particle being calculated, and the red lines indicate that the particle has the same state across different beads.}
		\label{fig:PIMC_grid_N=5}		
	\end{figure}

	\subsection{Energy estimator}
	
	The expectation value for energy can be calculated according to Eq. \eqref{eq:expectation_value} (or Eq. \eqref{eq:expectation_PIGS} for the case of PIGS)
	\begin{align}
		\label{eq:expectation_energy}
		\expval{\oper{E}}
		=& \frac{1}{\Z} \sum_{\m{}{1}}  \mel**{\m{}{1}}{e^{-\beta \H} \H}{\m{}{1}} \nn 
		=&\frac{1}{\Z}
		\sum_{\{\m{}{l}\}_{L+1}} \prod_{l=1}^{L}  \mel**{\m{}{l}}{ e^{ -\tau \H}}{\m{}{l+1}}  \mel{\m{}{L+1}}{\H}{\m{}{1}} \nn
		=& \frac{1}{\Z} 
		\sum_{\{\m{}{p}\}_{2L+1}}
		\prod_{p=1}^{2L}  \mel**{\m{}{p}}{\oper{\rho}(p)}{\m{}{p+1}}
		\mel{\m{}{2L+1}}{\H}{\m{}{1}}
		\nn 
		=& \frac{1}{\Z} 
		\sum_{\{\m{}{p}\}_{2L}}
		\sum_{\m{}{2L+1}}
		\prod_{p=1}^{2L-1}  \mel**{\m{}{p}}{\oper{\rho}(p)}{\m{}{p+1}}
		\mel**{\m{}{2L}}{e^{-\tau \V{even}}}{\m{}{2L+1}}
		\mel{\m{}{2L+1}}{\H}{\m{}{1}}
	\end{align}
	Here, the trick will be to multiply the term inside the sums by $\frac{\mel{\m{}{2L}}{e^{-\tau \V{even}}}{\m{}{1}}}{ \mel{\m{}{2L}}{e^{-\tau \V{even}}}{\m{}{1}}}$, so that
	\begin{align}
		\expval{\oper{E}}
		=& \frac{1}{\Z} 
		\sum_{\{\m{}{p}\}_{2L}}
		\sum_{\m{}{2L+1}}
		\prod_{p=1}^{2L-1}  \mel**{\m{}{p}}{\oper{\rho}(p)}{\m{}{p+1}}
		\mel**{\m{}{2L}}{e^{-\tau \V{even}}}{\m{}{2L+1}}
		\mel{\m{}{2L+1}}{\H}{\m{}{1}}
		{\color{red}
			\frac{\mel{\m{}{2L}}{e^{-\tau \V{even}}}{\m{}{1}}}{ \mel{\m{}{2L}}{e^{-\tau \V{even}}}{\m{}{1}}}
		}
		\nn
		=& \frac{1}{\Z} 
		\sum_{\{\m{}{p}\}_{2L}}
		\sum_{\m{}{2L+1}}
		\prod_{p=1}^{2L-1}  \mel**{\m{}{p}}{\oper{\rho}(p)}{\m{}{p+1}}
		{\color{red} 
			\mel{\m{}{2L}}{e^{-\tau \V{even}}}{\m{}{1}}
		}		
		\frac{
			\mel{\m{}{2L}}{e^{-\tau \V{even}}}{\m{}{2L+1}}
			\mel{\m{}{2L+1}}{\H}{\m{}{1}}
		}{
			{\color{red}
				\mel{\m{}{2L}}{e^{-\tau \V{even}}}{\m{}{1}}
			}		
		}			
		\nn
		=& \frac{1}{\Z} 
		\sum_{\{\m{}{p}\}_{2L}}
		\prod_{p=1}^{2L-1}  \mel**{\m{}{p}}{\oper{\rho}(p)}{\m{}{p+1}}
		\mel{\m{}{2L}}{e^{-\tau \V{even}}}{\m{}{1}}		
		\sum_{\m{}{2L+1}}	
		\frac{
			\mel{\m{}{2L}}{e^{-\tau \V{even}}}{\m{}{2L+1}}
		}{
			\mel{\m{}{2L}}{e^{-\tau \V{even}}}{\m{}{1}}		
		}
		\mel{\m{}{2L+1}}{\H}{\m{}{1}}
		\nn
		=& \frac{1}{\Z} 
		\sum_{\{\m{}{p}\}_{2L}}
		\prod_{p=1}^{2L-1}  \mel**{\m{}{p}}{\oper{\rho}(p)}{\m{}{p+1}}
		\mel{\m{}{2L}}{e^{-\tau \V{even}}}{\m{}{1}}		
		\sum_{\m{}{\alpha}}	
		\frac{
			\mel{\m{}{2L}}{e^{-\tau \V{even}}}{\m{}{\alpha}}
		}{
			\mel{\m{}{2L}}{e^{-\tau \V{even}}}{\m{}{1}}		
		}
		\mel{\m{}{\alpha}}{\H}{\m{}{1}}	\, .
	\end{align}
	Comparing the above result with Eq. \eqref{eq:partition_function_trotter_K+V}, we can write it as
	\begin{align}
		\expval{\oper{E}}
		=\sum_{\{\m{}{p}\}_{2L}} \mathcal{P}(\m{}{}) \mathcal{E}(\m{}{}) \, ,
	\end{align}
	where
	\begin{align}
		\mathcal{P}(\m{}{})
		=\frac{1}{\Z} \prod_{p=1}^{2L-1}  \mel**{\m{}{p}}{\oper{\rho}(p)}{\m{}{p+1}}
		\mel{\m{}{2L}}{e^{-\tau \H_{even}}}{\m{}{1}}
	\end{align}
	stands for the probability of some grid configuration $\m{}{} = (\m{}{1},\m{}{2}, \dots,\m{}{2L})$, and,
	\begin{align}
		\mathcal{E}(\m{}{})
		=\sum_{\m{}{\alpha}}	
		\frac{
			\mel{\m{}{2L}}{e^{-\tau \V{even}}}{\m{}{\alpha}}
		}{
			\mel{\m{}{2L}}{e^{-\tau \V{even}}}{\m{}{1}}		
		}
		\mel{\m{}{\alpha}}{\H}{\m{}{1}} \, ,
	\end{align} 
	is the energy calculated for such configuration, which can be written as
	\begin{align}
		\mathcal{E}
		=&\sum_{\m{}{\alpha}}	
		\frac{
			\mel{\m{}{2L}}{e^{-\tau \V{even}}}{\m{}{\alpha}}
		}{
			\mel{\m{}{2L}}{e^{-\tau \V{even}}}{\m{}{1}}		
		}
		\mel{\m{}{\alpha}}{\H}{\m{}{1}}
		\nn
%		=&\sum_{\m{}{\alpha}}	
%		\frac{
%			\mel{\m{}{2L}}{e^{-\tau \V{even}}}{\m{}{\alpha}}
%		}{
%			\mel{\m{}{2L}}{e^{-\tau \V{even}}}{\va{0}}		
%		}
%		\mel**{\m{}{\alpha}}{\V{~}}{\va{0}}
%		\nn
		=&\sum_{\m{1}{\alpha}}\sum_{\m{2}{\alpha}}\dots\sum_{\m{N}{\alpha}}\prod_{n \text{ even}}	
		\frac{
			\mel{\m{n}{2L},\m{n+1}{2L}}{e^{-\tau \V{n,n+1}}}{\m{n}{\alpha},\m{n+1}{\alpha}}
		}{
			\mel{\m{n}{2L},\m{n+1}{2L}}{e^{-\tau \V{n,n+1}}}{\m{i}{1},\m{i+1}{1}}		
		}
		\sum_{i=1}^{N-1}
		\mel{\m{i}{\alpha},\m{i+1}{\alpha}}{\H_{i,i+1}}{\m{i}{1},\m{i+1}{1}}
		\nn
		=&\sum_{\m{1}{\alpha}}\sum_{\m{2}{\alpha}}\dots\sum_{\m{N}{\alpha}}
		\prod_{n \text{ even}}	
		\frac{
			\mel{\m{n}{2L},\m{n+1}{2L}}{e^{-\tau \V{n,n+1}}}{\m{n}{\alpha},\m{n+1}{\alpha}}
		}{
			\mel{\m{n}{2L},\m{n+1}{2L}}{e^{-\tau \V{n,n+1}}}{0,0}		
		}
		\left(
		\sum_{i=1}^{N-1}
		\mel{\m{i}{\alpha},\m{i+1}{\alpha}}{\V{i,i+1}}{0,0}
		\right) \, .
	\end{align}

\newpage
\section{Problem with the estimator}
	
	We have our pseudo partition function given by
	\begin{align}
		\label{}
		\Z
		=& \mel**{\Psi_T}{e^{-\beta \H}}{\Psi_T}
		\nn 
		=&
		\sum_{\m{}{1}} \sum_{\m{}{2}} \dots \sum_{\m{}{L-1}}
		\mel**{\Psi_T}{e^{-\tau \H}}{\m{}{1}}
		\mel**{\m{}{1}}{e^{-\tau \H}}{\m{}{2}}
		\cdots		
		\mel**{\m{}{L-2}}{e^{-\tau \H}}{\m{}{L-1}}
		\mel**{\m{}{L-1}}{e^{-\tau \H}}{\Psi_T}
		\nn
		=&
		\sum_{\{\m{}{l}\}_{L-1}}
		\Pi\left(\Psi_T,\m{}{1},\dots, \m{}{L-1}, \Psi_T\right)
		\, ,
	\end{align}
	where $$\Pi\left(\Psi_T,\m{}{1},\dots, \m{}{L-1}, \Psi_T\right) \equiv \mel**{\Psi_T}{e^{-\tau \H}}{\m{}{1}}
	\mel**{\m{}{1}}{e^{-\tau \H}}{\m{}{2}}
	\cdots
	\mel**{\m{}{L-1}}{e^{-\tau \H}}{\Psi_T}$$ is the path contributing to $\Z$ and the sum $\sum_{\{\m{}{l}\}_{L-1}}$ ranges through all possible configurations of paths.Also the expectation value for the energy is given by
	\begin{align}
		\label{}
		\expval{\oper{E}}
		=& \frac{\mel**{\Psi_T}{e^{-\beta \H} \H}{\Psi_T}}{\mel**{\Psi_T}{e^{-\beta \H}}{\Psi_T}}
		\nn 
		=& \frac{1}{\Z} 
		\sum_{\m{}{1}} \dots \sum_{\m{}{L-1}} \sum_{\m{}{L}}
		\mel**{\Psi_T}{e^{-\tau \H}}{\m{}{1}}
		\cdots
		\mel**{\m{}{L-1}}{e^{-\tau \H}}{\m{}{L}}
		\mel**{\m{}{L}}{\H}{\Psi_T}
		\nn 
		=& \frac{1}{\Z} 
		\sum_{\{\m{}{l}\}_{L-1}} \sum_{\m{}{L}}
		\mel**{\Psi_T}{e^{-\tau \H}}{\m{}{1}}
		\cdots
		\mel**{\m{}{L-1}}{e^{-\tau \H}}{\m{}{L}}
		\mel**{\m{}{L}}{\H}{\Psi_T}
		{\color{red}
			\frac{\mel**{\m{}{L-1}}{e^{-\tau \H}}{\Psi_T}}{ \mel**{\m{}{L-1}}{e^{-\tau \H}}{\Psi_T}}
		}
		\nn 
		=& \frac{1}{\Z} 
		\sum_{\{\m{}{l}\}_{L-1}}
		\mel**{\Psi_T}{e^{-\tau \H}}{\m{}{1}}
		\cdots
		{\color{red}\mel**{\m{}{L-1}}{e^{-\tau \H}}{\Psi_T}}
		\frac{
			\sum_{\m{}{L}}
			\mel**{\m{}{L-1}}{e^{-\tau \H}}{\m{}{L}} 
			\mel**{\m{}{L}}{\H}{\Psi_T}
			}{
			{\color{red} \mel**{\m{}{L-1}}{e^{-\tau \H}}{\Psi_T}}
		}
		\nn 
		=& \frac{1}{\Z} 
		\sum_{\{\m{}{l}\}_{L-1}}
		\Pi\left(\Psi_T,\m{}{1},\dots, \m{}{L-1}, \Psi_T\right)
		\cdot
		\mathcal{E}\left(\m{}{L-1}, \Psi_T \right)
		\nn 
		=& \frac{\sum_{\{\m{}{l}\}_{L-1}}
			\Pi\left(\Psi_T,\m{}{1},\dots, \m{}{L-1}, \Psi_T\right)
			\cdot
			\mathcal{E}\left(\m{}{L-1}, \Psi_T \right)}{\sum_{\{\m{}{l}\}_{L-1}}
			\Pi\left(\Psi_T,\m{}{1},\dots, \m{}{L-1}, \Psi_T\right)} 
		\, ,
	\end{align}
	for $$\mathcal{E}\left(\m{}{L-1}, \Psi_T \right) \equiv \frac{
		\sum_{\m{}{L}}
		\mel**{\m{}{L-1}}{e^{-\tau \H}}{\m{}{L}} 
		\mel**{\m{}{L}}{\H}{\Psi_T}
	}{
		{\color{red} \mel**{\m{}{L-1}}{e^{-\tau \H}}{\Psi_T}}
	}$$
	representing the ``contribution on the energy of the path $\Pi$'', right?

	\subsection{Problem with the estimator N=3}
	
	We have our pseudo partition function given by
	\begin{align}
		\label{}
		\Z
		=& \mel**{\Psi_T}{e^{-\beta \H}}{\Psi_T}
		\nn 
		=&
		\sum_{\m{}{1}} \sum_{\m{}{2}} \dots \sum_{\m{}{L-1}}
		\mel**{\Psi_T}{e^{-\tau \H}}{\m{}{1}}
		\mel**{\m{}{1}}{e^{-\tau \H}}{\m{}{2}}
		\cdots		
		\mel**{\m{}{L-2}}{e^{-\tau \H}}{\m{}{L-1}}
		\mel**{\m{}{L-1}}{e^{-\tau \H}}{\Psi_T}
		\nn
		=&
		\sum_{\m{}{1}} \sum_{\m{}{2}} \dots \sum_{\m{}{L-1}}
		\mel**{\Psi_T}{e^{-\tau \K{}}e^{-\tau \V{}}}{\m{}{1}}
		\mel**{\m{}{1}}{e^{-\tau \K{}}e^{-\tau \V{}}}{\m{}{2}}
		\cdots		
		\mel**{\m{}{L-2}}{e^{-\tau \K{}}e^{-\tau \V{}}}{\m{}{L-1}}
		\mel**{\m{}{L-1}}{e^{-\tau \K{}}e^{-\tau \V{}}}{\Psi_T}
		\nn
		=&
		\sum_{\{\m{}{l}\}_{L-1}}
		\Pi\left(\Psi_T,\m{}{1},\dots, \m{}{L-1}, \Psi_T\right)
		\, ,
	\end{align}
	where $$\Pi\left(\Psi_T,\m{}{1},\dots, \m{}{L-1}, \Psi_T\right) \equiv \mel**{\Psi_T}{e^{-\tau \H}}{\m{}{1}}
	\mel**{\m{}{1}}{e^{-\tau \H}}{\m{}{2}}
	\cdots
	\mel**{\m{}{L-1}}{e^{-\tau \H}}{\Psi_T}$$ is the path contributing to $\Z$ and the sum $\sum_{\{\m{}{l}\}_{L-1}}$ ranges through all possible configurations of paths.Also the expectation value for the energy is given by
	\begin{align}
		\label{}
		\expval{\oper{E}}
		=& \frac{\mel**{\Psi_T}{e^{-\beta \H} \H}{\Psi_T}}{\mel**{\Psi_T}{e^{-\beta \H}}{\Psi_T}}
		\nn 
		=& \frac{1}{\Z} 
		\sum_{\m{}{1}} \dots \sum_{\m{}{L-1}} \sum_{\m{}{L}}
		\mel**{\Psi_T}{e^{-\tau \H}}{\m{}{1}}
		\cdots
		\mel**{\m{}{L-1}}{e^{-\tau \H}}{\m{}{L}}
		\mel**{\m{}{L}}{\H}{\Psi_T}
		\nn 
		=& \frac{1}{\Z} 
		\sum_{\{\m{}{l}\}_{L-1}} \sum_{\m{}{L}}
		\mel**{\Psi_T}{e^{-\tau \H}}{\m{}{1}}
		\cdots
		\mel**{\m{}{L-1}}{e^{-\tau \H}}{\m{}{L}}
		\mel**{\m{}{L}}{\H}{\Psi_T}
		{\color{red}
			\frac{\mel**{\m{}{L-1}}{e^{-\tau \H}}{\Psi_T}}{ \mel**{\m{}{L-1}}{e^{-\tau \H}}{\Psi_T}}
		}
		\nn 
		=& \frac{1}{\Z} 
		\sum_{\{\m{}{l}\}_{L-1}}
		\mel**{\Psi_T}{e^{-\tau \H}}{\m{}{1}}
		\cdots
		{\color{red}\mel**{\m{}{L-1}}{e^{-\tau \H}}{\Psi_T}}
		\frac{
			\sum_{\m{}{L}}
			\mel**{\m{}{L-1}}{e^{-\tau \H}}{\m{}{L}} 
			\mel**{\m{}{L}}{\H}{\Psi_T}
			}{
			{\color{red} \mel**{\m{}{L-1}}{e^{-\tau \H}}{\Psi_T}}
		}
		\nn 
		=& \frac{1}{\Z} 
		\sum_{\{\m{}{l}\}_{L-1}}
		\Pi\left(\Psi_T,\m{}{1},\dots, \m{}{L-1}, \Psi_T\right)
		\cdot
		\mathcal{E}\left(\m{}{L-1}, \Psi_T \right)
		\nn 
		=& \frac{\sum_{\{\m{}{l}\}_{L-1}}
			\Pi\left(\Psi_T,\m{}{1},\dots, \m{}{L-1}, \Psi_T\right)
			\cdot
			\mathcal{E}\left(\m{}{L-1}, \Psi_T \right)}{\sum_{\{\m{}{l}\}_{L-1}}
			\Pi\left(\Psi_T,\m{}{1},\dots, \m{}{L-1}, \Psi_T\right)} 
		\, ,
	\end{align}
	for $$\mathcal{E}\left(\m{}{L-1}, \Psi_T \right) \equiv \frac{
		\sum_{\m{}{L}}
		\mel**{\m{}{L-1}}{e^{-\tau \H}}{\m{}{L}} 
		\mel**{\m{}{L}}{\H}{\Psi_T}
	}{
		{\color{red} \mel**{\m{}{L-1}}{e^{-\tau \H}}{\Psi_T}}
	}$$
	representing the ``contribution on the energy of the path $\Pi$''
	
	However, when we multiply the numerator by the fraction in red, we can only do such trick if ${\color{red} \mel**{\m{}{L-1}}{e^{-\tau \H}}{\Psi_T}}$ is \textbf{non-zero} for all possible $\m{}{L-1}$'s. That guarantees us not to be summing an eventual $\frac{0}{0}$ term inside the sum on the numerator. What plays in our favour (or not) is the fact that \textbf{all} the updates of $\m{}{L-1}$ are done such that $\mel**{\m{}{L-1}}{e^{-\tau \H}}{\Psi_T} \neq 0$ because of the Gibbs Sampling!
	The curse here is that
	$${\mel**{\m{}{L-1}}{e^{-\tau \H}}{\Psi_T}} \neq 0 \implies \Pi\left(\Psi_T,\m{}{1},\dots, \m{}{L-1}, \Psi_T\right) \neq 0$$
	necessarily. So, because $\Z$ is actually what we sample and we are limiting ourselves to sample only the non-zero contributions, we are sometimes leaving behind the $\m{}{L-1}$'s that do make $\Pi\left(\Psi_T,\m{}{1},\dots, \m{}{L-1}, \Psi_T\right) = 0$ but not necessarily the term
	$$
	\sum_{\m{}{L}}\Pi\left(\Psi_T,\m{}{1},\dots, \m{}{L-1}, \m{}{L}\right) \cdot 
	\mel**{\m{}{L}}{\H}{\Psi_T}
	$$
	which could be contributing to the numerator on the $\expval{\oper{E}}$ equation. That is what I meant when I said that sometimes the paths contributing to $\Z$ are not the same as the ones contributing to the numerator on the $\expval{\oper{E}}$. Does that make sense?
	
	Here the contribution of the last beads for $\Z$, when $N=3$ and $\ket{\Psi_T} = \ket{0,0,0}$, is
	\begin{align}
		\Pi\left(\m{}{L-1},\Psi_T\right) 
		&= \mel**{\m{}{L-1}}{e^{-\tau \H}}{\Psi_T}
		\nn
		&= \mel**{\m{}{L-1}}{e^{-\tau \K{}}e^{-\tau \V{}}}{0,0,0}
		\nn
		&= e^{-\tau {\m{}{L-1}}^2} \mel**{\m{}{L-1}}{e^{-\tau \V{12}}e^{-\tau \V{23}}}{0,0,0}
		\nn
		&= \sum_{\m{}{\alpha}}e^{-\tau {\m{}{L-1}}^2} \mel**{\m{}{L-1}}{e^{-\tau \V{12}}}{\m{}{\alpha}}\mel**{\m{}{\alpha}}{e^{-\tau \V{23}}}{0,0,0}
		\nn
		&=  \sum_{\m{1}{\alpha}}\sum_{\m{2}{\alpha}}\sum_{\m{3}{\alpha}}
		e^{-\tau {\m{1}{L-1}}^2}e^{-\tau {\m{2}{L-1}}^2}e^{-\tau {\m{3}{L-1}}^2} 
		\mel**{\m{1}{L-1},\m{2}{L-1}}{e^{-\tau \V{12}}}{\m{1}{\alpha},\m{2}{\alpha}} \d{\m{3}{L-1},\m{3}{\alpha}}
		\times \nn
		&\times
		\d{\m{1}{\alpha},0}
		\mel**{\m{2}{\alpha},\m{3}{\alpha}}{e^{-\tau \V{23}}}{0,0}
		\nn
		&=  \sum_{\m{2}{\alpha}}
		e^{-\tau {\m{1}{L-1}}^2}e^{-\tau {\m{2}{L-1}}^2}e^{-\tau {\m{3}{L-1}}^2} 
		\mel**{\m{1}{L-1},\m{2}{L-1}}{e^{-\tau \V{12}}}{0,\m{2}{\alpha}}
		\mel**{\m{2}{\alpha},\m{3}{L-1}}{e^{-\tau \V{23}}}{0,0}
		\, ,
	\end{align}	
	and after relabelling the beads we have 
	\begin{align}
		\Pi\left(\m{}{P-2},\Psi_T\right) 
		=  \sum_{\m{2}{P-1}}
		e^{-\tau \left(\m{}{P-2}\right)^2} 
		\mel**{\m{1}{P-2},\m{2}{P-2}}{e^{-\tau \V{12}}}{0,\m{2}{P-1}}
		\mel**{\m{2}{P-1},\m{3}{P-2}}{e^{-\tau \V{23}}}{0,0}
		\, .
	\end{align}
	The contribution for the energy is
	\begin{align}
		&\sum_{\m{}{L}}\Pi\left(\m{}{L-1}, \m{}{L}\right) \cdot 
		\mel**{\m{}{L}}{\H}{\Psi_T} =		
		\nn
		&= \mel**{\m{}{L-1}}{e^{-\tau \H} \H}{\Psi_T}
		\nn
		&= \mel**{\m{}{L-1}}{e^{-\tau \K{}}e^{-\tau \V{}} \left(\K{} + \V{}\right)}{0,0,0}
		\nn
		&= e^{-\tau {\m{}{L-1}}^2}
		\mel**{\m{}{L-1}}{e^{-\tau \V{12}}e^{-\tau \V{23}} \left(\V{12} + \V{23}\right)}{0,0,0}
		\nn
		&= e^{-\tau {\m{}{L-1}}^2}
		\mel**{\m{}{L-1}}{e^{-\tau \V{12}}\V{12}~e^{-\tau \V{23}}}{0,0,0}
		+
		e^{-\tau {\m{}{L-1}}^2}
		\mel**{\m{}{L-1}}{e^{-\tau \V{12}}~e^{-\tau \V{23}}\V{23}}{0,0,0}
		\nn
		&= \sum_{\m{}{L}}
		e^{-\tau {\m{}{L-1}}^2}
		\mel**{\m{}{L-1}}{e^{-\tau \V{12}}\V{12}
		\ketbra{\m{}{L}}
		e^{-\tau \V{23}}}{0,0,0}
		+
		\sum_{\m{}{L}}
		e^{-\tau {\m{}{L-1}}^2}
		\mel**{\m{}{L-1}}{e^{-\tau \V{12}}			
		\ketbra{\m{}{L}}
		e^{-\tau \V{23}}\V{23}}{0,0,0}
		\nn
		&= \sum_{\m{}{L}}
		e^{-\tau {\m{}{L-1}}^2}
		\mel**{\m{1}{L-1},\m{2}{L-1}}{e^{-\tau \V{12}}\V{12}
		}{\m{1}{L},\m{2}{L}} \d{\m{3}{L-1},\m{3}{L}}~
		\d{\m{1}{L},0}\mel**{\m{2}{L},\m{3}{L}}{e^{-\tau \V{23}}}{0,0}
		+ \nn
		&+
		\sum_{\m{}{L}}
		e^{-\tau {\m{}{L-1}}^2}
		\mel**{\m{1}{L-1},\m{2}{L-1}}{e^{-\tau \V{12}}
		}{\m{1}{L},\m{2}{L}} \d{\m{3}{L-1},\m{3}{L}}~
		\d{\m{1}{L},0}\mel**{\m{2}{L},\m{3}{L}}{e^{-\tau \V{23}}\V{23}}{0,0}
		\nn
		&= \sum_{\m{2}{L}}
		e^{-\tau \left(\m{1}{L-1}\right)^2}e^{-\tau \left(\m{2}{L-1}\right)^2}e^{-\tau \left(\m{3}{L-1}\right)^2}
		\mel**{\m{1}{L-1},\m{2}{L-1}}{e^{-\tau \V{12}}\V{12}
		}{0,\m{2}{L}} 
		\mel**{\m{2}{L},\m{3}{L-1}}{e^{-\tau \V{23}}}{0,0}
		+ \nn
		&+
		\sum_{\m{2}{L}}
		e^{-\tau \left(\m{1}{L-1}\right)^2}e^{-\tau \left(\m{2}{L-1}\right)^2}e^{-\tau \left(\m{3}{L-1}\right)^2}
		\mel**{\m{1}{L-1},\m{2}{L-1}}{e^{-\tau \V{12}}
		}{0,\m{2}{L}} 
		\mel**{\m{2}{L},\m{3}{L-1}}{e^{-\tau \V{23}}\V{23}}{0,0}
		\, ,
	\end{align}
	and again relabelling the beads we have
	\begin{align}
		&\sum_{\m{}{\alpha}}\mel**{\m{}{P-1}}{e^{-\tau \H}} {\m{}{\alpha}}\mel**{\m{}{\alpha}}{\H}{\Psi_T} =		
		\nn
		&= \sum_{\m{2}{\alpha}}
		e^{-\tau \left(\m{1}{P-1}\right)^2}e^{-\tau \left(\m{2}{P-1}\right)^2}e^{-\tau \left(\m{3}{P-1}\right)^2}
		\mel**{\m{1}{P-1},\m{2}{P-1}}{e^{-\tau \V{12}}\V{12}
		}{0,\m{2}{\alpha}} 
		\mel**{\m{2}{\alpha},\m{3}{P-1}}{e^{-\tau \V{23}}}{0,0}
		+ \nn
		&+
		\sum_{\m{2}{\alpha}}
		e^{-\tau \left(\m{1}{P-1}\right)^2}e^{-\tau \left(\m{2}{P-1}\right)^2}e^{-\tau \left(\m{3}{P-1}\right)^2}
		\mel**{\m{1}{P-1},\m{2}{P-1}}{e^{-\tau \V{12}}
		}{0,\m{2}{\alpha}} 
		\mel**{\m{2}{\alpha},\m{3}{P-1}}{e^{-\tau \V{23}}\V{23}}{0,0}
		\, ,
	\end{align}

\subsection{$N>3$}
	Here the contribution of the last beads for $\Z$, when $N>3$ and $\ket{\Psi_T} = \ket{0,0,0}$, is
	\begin{align}
		\Pi\left(\m{}{P-2},\Psi_T\right) 
		=& \mel**{\m{}{P-2}}{e^{-\tau \H}}{\Psi_T}
		\nn
		=& \mel**{\m{}{P-2}}{e^{-\tau \K{}}e^{-\tau \V{}}}{\va{0}}
		\nn
		=& e^{-\tau \left(\m{}{P-2}\right)^2}  \mel**{\m{}{P-2}}{e^{-\tau \V{odd}}e^{-\tau \V{even}}}{\va{0}}
		\nn
		=& \sum_{\m{}{P-1}}
		e^{-\tau \left(\m{}{P-2}\right)^2}  \mel**{\m{}{P-2}}{e^{-\tau \V{12}}e^{-\tau \V{34}}\dots}{\m{}{P-1}}\mel**{\m{}{P-1}}{e^{-\tau \V{23}}e^{-\tau \V{45}}\dots}{\va{0}}
		\nn
		=& \sum_{\m{}{P-1}}
		e^{-\tau \left(\m{}{P-2}\right)^2}  \mel**{\m{1}{P-2},\m{2}{P-2}}{e^{-\tau \V{12}}}{\m{1}{P-1},\m{2}{P-1}}
		\mel**{\m{3}{P-2},\m{4}{P-2}}{e^{-\tau \V{34}}}{\m{3}{P-1},\m{4}{P-1}}
		\cdots
		\times \nn
		& \times 
		\mel**{\m{2}{P-1},\m{3}{P-1}}{e^{-\tau \V{23}}}{0,0}
		\mel**{\m{4}{P-1},\m{5}{P-1}}{e^{-\tau \V{45}}}{0,0}
		\cdots
		\nn
		=& \sum_{\m{}{P-1}}
		e^{-\tau \left(\m{}{P-2}\right)^2}  
		\prod_{i \text{ odd}}\mel**{\m{i}{P-2},\m{i+1}{P-2}}{e^{-\tau \V{i,i+1}}}{\m{i}{P-1},\m{i+1}{P-1}}
		\prod_{j \text{ even}}\mel**{\m{j}{P-1},\m{j+1}{P-1}}{e^{-\tau \V{j,j+1}}}{0,0}
		\, ,
	\end{align}	
	and after relabelling the beads we have 
	\begin{align}
		\Pi\left(\m{}{P-2},\Psi_T\right) 
		=  \sum_{\m{2}{P-1}}
		e^{-\tau \left(\m{}{P-2}\right)^2} 
		\mel**{\m{1}{P-2},\m{2}{P-2}}{e^{-\tau \V{12}}}{0,\m{2}{P-1}}
		\mel**{\m{2}{P-1},\m{3}{P-2}}{e^{-\tau \V{23}}}{0,0}
		\, .
	\end{align}
	The contribution for the energy is
	\begin{align}
		&\sum_{\m{}{L}}\Pi\left(\m{}{L-1}, \m{}{L}\right) \cdot 
		\mel**{\m{}{L}}{\H}{\Psi_T} =		
		\nn
		&= \mel**{\m{}{L-1}}{e^{-\tau \H} \H}{\Psi_T}
		\nn
		&= \mel**{\m{}{L-1}}{e^{-\tau \K{}}e^{-\tau \V{}} \left(\K{} + \V{}\right)}{0,0,0}
		\nn
		&= e^{-\tau {\m{}{L-1}}^2}
		\mel**{\m{}{L-1}}{e^{-\tau \V{12}}e^{-\tau \V{23}} \left(\V{12} + \V{23}\right)}{0,0,0}
		\nn
		&= e^{-\tau {\m{}{L-1}}^2}
		\mel**{\m{}{L-1}}{e^{-\tau \V{12}}\V{12}~e^{-\tau \V{23}}}{0,0,0}
		+
		e^{-\tau {\m{}{L-1}}^2}
		\mel**{\m{}{L-1}}{e^{-\tau \V{12}}~e^{-\tau \V{23}}\V{23}}{0,0,0}
		\nn
		&= \sum_{\m{}{L}}
		e^{-\tau {\m{}{L-1}}^2}
		\mel**{\m{}{L-1}}{e^{-\tau \V{12}}\V{12}
			\ketbra{\m{}{L}}
			e^{-\tau \V{23}}}{0,0,0}
		+
		\sum_{\m{}{L}}
		e^{-\tau {\m{}{L-1}}^2}
		\mel**{\m{}{L-1}}{e^{-\tau \V{12}}			
			\ketbra{\m{}{L}}
			e^{-\tau \V{23}}\V{23}}{0,0,0}
		\nn
		&= \sum_{\m{}{L}}
		e^{-\tau {\m{}{L-1}}^2}
		\mel**{\m{1}{L-1},\m{2}{L-1}}{e^{-\tau \V{12}}\V{12}
		}{\m{1}{L},\m{2}{L}} \d{\m{3}{L-1},\m{3}{L}}~
		\d{\m{1}{L},0}\mel**{\m{2}{L},\m{3}{L}}{e^{-\tau \V{23}}}{0,0}
		+ \nn
		&+
		\sum_{\m{}{L}}
		e^{-\tau {\m{}{L-1}}^2}
		\mel**{\m{1}{L-1},\m{2}{L-1}}{e^{-\tau \V{12}}
		}{\m{1}{L},\m{2}{L}} \d{\m{3}{L-1},\m{3}{L}}~
		\d{\m{1}{L},0}\mel**{\m{2}{L},\m{3}{L}}{e^{-\tau \V{23}}\V{23}}{0,0}
		\nn
		&= \sum_{\m{2}{L}}
		e^{-\tau \left(\m{1}{L-1}\right)^2}e^{-\tau \left(\m{2}{L-1}\right)^2}e^{-\tau \left(\m{3}{L-1}\right)^2}
		\mel**{\m{1}{L-1},\m{2}{L-1}}{e^{-\tau \V{12}}\V{12}
		}{0,\m{2}{L}} 
		\mel**{\m{2}{L},\m{3}{L-1}}{e^{-\tau \V{23}}}{0,0}
		+ \nn
		&+
		\sum_{\m{2}{L}}
		e^{-\tau \left(\m{1}{L-1}\right)^2}e^{-\tau \left(\m{2}{L-1}\right)^2}e^{-\tau \left(\m{3}{L-1}\right)^2}
		\mel**{\m{1}{L-1},\m{2}{L-1}}{e^{-\tau \V{12}}
		}{0,\m{2}{L}} 
		\mel**{\m{2}{L},\m{3}{L-1}}{e^{-\tau \V{23}}\V{23}}{0,0}
		\, ,
	\end{align}
	and again relabelling the beads we have
	\begin{align}
		&\sum_{\m{}{\alpha}}\mel**{\m{}{P-1}}{e^{-\tau \H}} {\m{}{\alpha}}\mel**{\m{}{\alpha}}{\H}{\Psi_T} =		
		\nn
		&= \sum_{\m{2}{\alpha}}
		e^{-\tau \left(\m{1}{P-1}\right)^2}e^{-\tau \left(\m{2}{P-1}\right)^2}e^{-\tau \left(\m{3}{P-1}\right)^2}
		\mel**{\m{1}{P-1},\m{2}{P-1}}{e^{-\tau \V{12}}\V{12}
		}{0,\m{2}{\alpha}} 
		\mel**{\m{2}{\alpha},\m{3}{P-1}}{e^{-\tau \V{23}}}{0,0}
		+ \nn
		&+
		\sum_{\m{2}{\alpha}}
		e^{-\tau \left(\m{1}{P-1}\right)^2}e^{-\tau \left(\m{2}{P-1}\right)^2}e^{-\tau \left(\m{3}{P-1}\right)^2}
		\mel**{\m{1}{P-1},\m{2}{P-1}}{e^{-\tau \V{12}}
		}{0,\m{2}{\alpha}} 
		\mel**{\m{2}{\alpha},\m{3}{P-1}}{e^{-\tau \V{23}}\V{23}}{0,0}
		\, ,
	\end{align}

\newpage
	\begin{align}
		&\sum_{\m{}{L}}\Pi\left(\m{}{L-1}, \m{}{L}\right) \cdot 
		\mel**{\m{}{L}}{\H}{\Psi_T} =		
		\nn
		&= \sum_{\m{}{L}}\mel**{\m{}{L-1}}{e^{-\tau \H}}{\m{}{L}}\mel**{\m{}{L}}{\H}{\Psi_T}
		\nn
		&= \sum_{\m{}{L}}\mel**{\m{}{L-1}}{e^{-\tau \K{}}e^{-\tau \V{}}}{\m{}{L}}\mel**{\m{}{L}}{\H}{0,0,0}
		\nn
		&= \sum_{\m{}{L}}e^{-\tau {\m{}{L-1}}^2} \mel**{\m{}{L-1}}{e^{-\tau \V{12}}e^{-\tau \V{23}}}{\m{}{L}}\mel**{\m{}{L}}{\H}{0,0,0}
		\nn
		&= \sum_{\m{}{L}}\sum_{\m{}{\alpha}}e^{-\tau {\m{}{L-1}}^2} \mel**{\m{}{L-1}}{e^{-\tau \V{12}}}{\m{}{\alpha}}\mel**{\m{}{\alpha}}{e^{-\tau \V{23}}}{\m{}{L}}\mel**{\m{}{L}}{\H}{0,0,0}
		\nn
		&= 
		\sum_{\m{1}{L}}\sum_{\m{2}{L}}\sum_{\m{3}{L}} \sum_{\m{1}{\alpha}}\sum_{\m{2}{\alpha}}\sum_{\m{3}{\alpha}}
		e^{-\tau {\m{}{L-1}}^2}
		\mel**{\m{1}{L-1},\m{2}{L-1}}{e^{-\tau \V{12}}}{\m{1}{\alpha},\m{2}{\alpha}} \d{\m{3}{L-1},\m{3}{\alpha}}
		\times \nn
		&\times
		\d{\m{1}{\alpha},\m{1}{L}}
		\mel**{\m{2}{\alpha},\m{3}{\alpha}}{e^{-\tau \V{23}}}{\m{2}{L},\m{3}{L}}\mel**{\m{1}{L},\m{2}{L},\m{3}{L}}{(\K{} + \V{})}{0,0,0}
		\nn
		&= 
		\sum_{\m{1}{L}}\sum_{\m{2}{L}}\sum_{\m{3}{L}} \sum_{\m{2}{\alpha}}
		e^{-\tau {\m{}{L-1}}^2} 
		\mel**{\m{1}{L-1},\m{2}{L-1}}{e^{-\tau \V{12}}}{\m{1}{L},\m{2}{\alpha}}\mel**{\m{2}{\alpha},\m{3}{L-1}}{e^{-\tau \V{23}}}{\m{2}{L},\m{3}{L}}
		\times \nn
		&\times
		\mel**{\m{1}{L},\m{2}{L},\m{3}{L}}{(\V{12} + \V{23})}{0,0,0}
		\nn
		&= 
		\sum_{\m{1}{L}}\sum_{\m{2}{L}}\sum_{\m{3}{L}} \sum_{\m{2}{\alpha}}
		e^{-\tau {\m{}{L-1}}^2}
		\mel**{\m{1}{L-1},\m{2}{L-1}}{e^{-\tau \V{12}}}{\m{1}{L},\m{2}{\alpha}}\mel**{\m{2}{\alpha},\m{3}{L-1}}{e^{-\tau \V{23}}}{\m{2}{L},\m{3}{L}}
		\times \nn
		&\times
		\left(\mel**{\m{1}{L},\m{2}{L}}{\V{12}}{0,0}\d{\m{3}{L},0} + \d{\m{1}{L},0}\mel**{\m{2}{L},\m{3}{L}}{\V{23}}{0,0}\right)
		\nn
		&= 
		\sum_{\m{1}{L}}\sum_{\m{2}{L}} \sum_{\m{2}{\alpha}}
		e^{-\tau {\m{}{L-1}}^2}
		\mel**{\m{1}{L-1},\m{2}{L-1}}{e^{-\tau \V{12}}}{\m{1}{L},\m{2}{\alpha}}\mel**{\m{2}{\alpha},\m{3}{L-1}}{e^{-\tau \V{23}}}{\m{2}{L},0}
		\mel**{\m{1}{L},\m{2}{L}}{\V{12}}{0,0}
		+ \nn
		&+
		\sum_{\m{2}{L}}\sum_{\m{3}{L}} \sum_{\m{2}{\alpha}}
		e^{-\tau {\m{}{L-1}}^2}
		\mel**{\m{1}{L-1},\m{2}{L-1}}{e^{-\tau \V{12}}}{0,\m{2}{\alpha}}\mel**{\m{2}{\alpha},\m{3}{L-1}}{e^{-\tau \V{23}}}{\m{2}{L},\m{3}{L}}
		\mel**{\m{2}{L},\m{3}{L}}{\V{23}}{0,0}
		\, ,
	\end{align}	
	and after relabelling the beads we have 
	\begin{align}
		&\sum_{\m{}{L}}\Pi\left(\m{}{L-1}, \m{}{L}\right) \cdot 
		\mel**{\m{}{L}}{\H}{\Psi_T} =
		= 
		\sum_{\m{1}{L}}\sum_{\m{2}{L}} \sum_{\m{2}{\alpha}}
		e^{-\tau {\m{}{L-1}}^2}
		\mel**{\m{1}{L-1},\m{2}{L-1}}{e^{-\tau \V{12}}}{\m{1}{L},\m{2}{\alpha}}\mel**{\m{2}{\alpha},\m{3}{L-1}}{e^{-\tau \V{23}}}{\m{2}{L},0}
		\mel**{\m{1}{L},\m{2}{L}}{\V{12}}{0,0}
		+ \nn
		&+
		\sum_{\m{2}{L}}\sum_{\m{3}{L}} \sum_{\m{2}{\alpha}}
		e^{-\tau {\m{}{L-1}}^2}
		\mel**{\m{1}{L-1},\m{2}{L-1}}{e^{-\tau \V{12}}}{0,\m{2}{\alpha}}\mel**{\m{2}{\alpha},\m{3}{L-1}}{e^{-\tau \V{23}}}{\m{2}{L},\m{3}{L}}
		\mel**{\m{2}{L},\m{3}{L}}{\V{23}}{0,0}
		\, ,
	\end{align}
	
\newpage
%\section{Excluded parts}
%
%\newpage
\subsection{Trotter Expansion $\H = \H_{odd} + \H_{even}$}

The hamiltonian can be represented as
\begin{align}
	\H 
	=& \sum_i \K{i} + \sum_{i} \V{i,i+1} \nn
	=& \frac{\K{1}}{2} + \sum_{i=1}^N \frac{\K{i}+\K{i+1}}{2} + \frac{\K{N}}{2} + \sum_{i=1}^N \V{i,i+1} \nn
	=& \frac{\K{1}}{2} + \sum_{i=1}^{N-1} \frac{\K{i}+\K{i+1}}{2}+ \V{i,i+1} + \frac{\K{N}}{2} \nn
	=& \frac{\K{1}}{2} + \frac{\K{N}}{2} + \sum_{i=1}^{N-1} \H_{i,i+1} \nn
\end{align}

Here, it is convenient to split the sum into only two terms as
\begin{equation}
	\H = \frac{\K{1}}{2} + \frac{\K{N}}{2} + \H_{odd} + \H_{even} 
	\text{ , for }
	\left\{
	\begin{aligned}
		&\H_{odd} \equiv \H_{12} + \H_{34} + \dots =  \sum_{n \text{ odd}}^{N-1} \H_{n,n+1} \, ,\\
		&\H_{even} \equiv \H_{23} + \H_{45} + \dots =\sum_{n \text{ even}}^{N-1} \H_{n,n+1}  \, .
	\end{aligned}
	\right.
\end{equation}
It is important to notice that $\comm{\H_{odd}}{\H_{even}} \neq 0$. After applying the \textbf{Trotter expansion}, Eq. \eqref{eq:partition_function_PIMC} becomes
\begin{align}
	\label{eq:partition_function_PIMC_trotter_Heven_Hodd}
	\Z(\beta) 
	\approx& \sum_{\{\m{}{l}\}_{L}} \prod_{l=1}^{L} \mel**{\m{}{l}}{e^{-\tau \frac{\K{1}+K{N}}{2}}e^{-\tau \H_{odd}}e^{-\tau \H_{even}}}{\m{}{l+1}} \nn
	=& \sum_{\{\m{}{l}\}_{L}} \prod_{l=1}^{L} e^{ \frac{-\tau}{2}(\m{1}{l}+\m{N}{l})} \mel{\m{}{l}}{ e^{ -\tau \H_{odd}} {\color{red} \sum_{\m{}{\alpha}} \ketbra{\m{}{\alpha}}} e^{ -\tau \H_{even}}}{\m{}{l+1}}\nn
	=& \sum_{\{\m{}{l}\}_{L}} \prod_{l=1}^{L} \rho_{K_{1N}}^l \sum_{\m{}{\alpha}} \mel**{\m{}{l}}{ e^{ -\tau \H_{odd}}}{\m{}{\alpha}} \mel**{\m{}{\alpha}}{e^{ -\tau \H_{even}}}{\m{}{l+1}}
	\nn
	=& \sum_{\{\m{}{l}\}_{L}}
	\rho_{K_{1N}}^1 \left(\sum_{\m{}{\alpha\sub{1}}}   \mel{\m{}{1}}{ e^{ -\tau \H_{odd}}}{\m{}{\alpha\sub{1}}} \mel{\m{}{\alpha\sub{1}}}{e^{ -\tau \H_{even}}}{\m{}{2}}\right) \times \nn
	& \hspace{0.55cm} \times
	\rho_{K_{1N}}^2 \left(\sum_{\m{}{\alpha\sub{2}}}  \mel{\m{}{2}}{ e^{ -\tau \H_{odd}}}{\m{}{\alpha\sub{2}}} \mel{\m{}{\alpha\sub{2}}}{e^{ -\tau \H_{even}}}{\m{}{3}}\right) \times \dots \nn		
	& \hspace{0.08cm} \dots \times
	\rho_{K_{1N}}^L \left(\sum_{\m{}{\alpha\sub{L}}} \mel{\m{}{L}}{ e^{ -\tau \H_{odd}}}{\m{}{\alpha\sub{L}}} \mel{\m{}{\alpha\sub{L}}}{e^{ -\tau \H_{even}}}{\m{}{L+1}}\right)
	\nn
	%
	=& \sum_{\m{}{1}} \sum_{\m{}{2}} \sum_{\m{}{3}}\dots  \sum_{\m{}{L}} ~~
	\sum_{\m{}{\alpha\sub{1}}} \sum_{\m{}{\alpha\sub{2}}} \sum_{\m{}{\alpha\sub{3}}} \dots \sum_{\m{}{\alpha\sub{L}}} 
	\prod_{l=1}^{L} \rho_{K_{1N}}^l \mel**{\m{}{l}}{e^{ -\tau \H_{odd}}}{\m{}{\alpha\sub{l}}}
	\mel**{\m{}{\alpha\sub{l}}}{e^{ -\tau \H_{even}}}{\m{}{l+1}}
	\, ,
\end{align} 
Then, reorganizing the sums and relabelling the terms as $l \rightarrow 2p-1$, $\alpha\sub{l} \rightarrow 2p$, and $l+1 \rightarrow 2p+1$ we have
\begin{align}
	\Z(\beta)
	=& \sum_{\{\m{}{p}\}_{2L}}
	\prod_{p=1}^{L} \rho_{K_{1N}}^l\mel**{\m{}{2p-1}}{e^{ -\tau \H_{odd}}}{\m{}{2p}}
	\mel**{\m{}{2p}}{e^{ -\tau \H_{even}}}{\m{}{2p+1}} \nn
	\Z(\beta)=& \sum_{\{\m{}{p}\}_{2L}}
	\prod_{p=1}^{2L}  \mel**{\m{}{p}}{\oper{\rho}(p)}{\m{}{p+1}}
	\, ,
\end{align}
where
\begin{equation}
	\label{eq:def_rho_tau_op}
	\oper{\rho}(p) =
	\left\{
	\begin{aligned}
		&\rho_{K_{1N}}^p e^{ -\tau \H_{odd}} & \text{, for $p$ odd,} \\
		&e^{ -\tau \H_{even}} & \text{, for $p$ even.} 			
	\end{aligned}
	\right.
\end{equation}
Calculating each term of the matrix elements we have,
\begin{align}
	\mel**{\m{}{p}}{e^{ -\tau \H_{odd}}}{\m{}{p+1}}
	=& \mel**{\m{}{p}}{e^{ -\tau \sum_{n \text{ odd}}^{N-1} \H_{n,n+1}}}{\m{}{p+1}} \nn
	=& \mel{\m{}{p}}{\prod_{n \text{ odd}}^{N-1} e^{ -\tau \H_{n,n+1}} }{\m{}{p+1}}\nn
	=& \prod_{n \text{ odd}}^{N-1} \mel**{\m{n}{p},\m{n+1}{p}}{e^{ -\tau \H_{n,n+1}}}{\m{n}{p+1},\m{n+1}{p+1}} \, ,
\end{align}
where we have omitted all the delta functions regarding the other states to keep the notation simplified. Also, analogously
\begin{align}
	\mel**{\m{}{p}}{e^{ -\tau \H_{even}}}{\m{}{p+1}}
	=& \prod_{n \text{ even}}^{N-1}\mel{\m{n}{p},\m{n+1}{p}}{e^{ -\tau \H_{n,n+1}}}{\m{n}{p+1},\m{n+1}{p+1}}  \, .
\end{align}
Then, the partition function becomes
\begin{align}
	\label{eq:part_function_expanded}
	\boxed{
		\mathcolorbox{highlight}{
			\Z(\beta)
			= \sum_{\{\m{}{p}\}_{2L}}
			\prod_{p=1}^{2L} \prod_{n \in \mathcal{A}_p} \mel{\m{n}{p},\m{n+1}{p}}{}{\m{n}{p+1},\m{n+1}{p+1}}
			\, ,
	}}
\end{align}
for set $\mathcal{A}_p$ off all odd (even) numbers in the interval $\{1,N-1\}$ if $p$ is an odd (even) number, and where, now, the double bar notation $||$ makes implicit the matrix element being taken with respect to the $\oper{\rho}(p)$ as defined on eq. \eqref{eq:def_rho_tau_op}.




%	\subsection{Gibbs Sampling}
%		\begin{align}
	%		\vec{M}(t) 
	%		&= \left[\m{}{1},\m{}{2}, \dots \m{}{P}\right] \nonumber \\
	%		&= \left[\m{1}{1},\m{2}{1}, \dots \m{N}{1},\dots ,\m{n}{p}, \dots, \m{1}{P}, \m{2}{P}, \dots \m{N}{P}\right]\nonumber
	%	\end{align}
%	$$$$
%	
%	$$\text{Pr}\left({\color{ForestGreen}\m{n}{p}} | {\color{RoyalPurple}\m{1}{1}},\dots,{\color{RoyalPurple}\m{1}{1}},\dots ,{\color{RoyalPurple}\m{n-1}{p}},{\color{ForestGreen}\square}, \m{n+1}{p} , \dots, \m{1}{P}, \dots \m{N}{P}\right) \hspace{1cm}
%	$$
%	$$\hspace{1cm}=\dfrac{
	%		\text{Pr}\left({\color{RoyalPurple}\m{1}{1}},\dots,{\color{RoyalPurple}\m{1}{1}},\dots ,{\color{RoyalPurple}\m{n-1}{p}},{\color{ForestGreen}\m{n}{p}}, \m{n+1}{p} , \dots, \m{1}{P}, \dots \m{N}{P}\right)
	%	}{
	%		\sum_{{\color{ForestGreen}\m{n}{p}}} \text{Pr}\left({\color{RoyalPurple}\m{1}{1}},\dots,{\color{RoyalPurple}\m{1}{1}},\dots ,{\color{RoyalPurple}\m{n-1}{p}},{\color{ForestGreen}\m{n}{p}}, \m{n+1}{p} , \dots, \m{1}{P}, \dots \m{N}{P}\right)}
%	$$
%	
%	$$\text{Pr}\left({\color{ForestGreen} m_{t}} | m_1,m_2,\dots,m_6 \right)
%	=
%	\frac{
	%		\mel**{m_1,m_2}{e^{-\tau \V{}}}{m_3,{\color{ForestGreen} m_{t}}}
	%		\mel**{{\color{ForestGreen} m_t},m_4}{e^{-\tau \V{}}}{m_5,m_6}
	%	}{
	%		\sum_{{\color{ForestGreen} m_t}} \mel**{m_1,m_2}{e^{-\tau \V{}}}{m_3,{\color{ForestGreen} m_{t}}}
	%		\mel**{{\color{ForestGreen} m_t},m_4}{e^{-\tau \V{}}}{m_5,m_6}}
%	$$

\subsubsection{Energy Estimator for PIGS}
\begin{align}
	\label{eq:expectation_energy_test3}
	\expval{\oper{E}}
	=& \frac{ \mel**{\Psi_T}{e^{-\frac{\beta}{2} \H} \H e^{- \frac{\beta}{2} \H }}{\Psi_T}}{\mel**{\Psi_T}{e^{-\beta \H}}{\Psi_T}} \nn 
	=& \frac{1}{\Z} \mel**{\Psi_T}{e^{-\beta \H} \H }{\Psi_T} \nn 
	=&\frac{1}{\Z}
	\sum_{\{\m{}{l}\}_{L+1}} \delta_{\Psi_T,\m{}{1}} \prod_{l=1}^{L}  \mel**{\m{}{l}}{ e^{ -\tau \H}}{\m{}{l+1}}  \mel{\m{}{L+1}}{\H}{\Psi_T}  \nn
	\approx& \frac{1}{\Z}
	\sum_{\{\m{}{l}\}_{L+1}} \delta_{\Psi_T,\m{}{1}} \prod_{l=1}^{L}  \mel**{\m{}{l}}{ e^{ -\tau \H_{odd}} e^{ -\tau \H_{even}}}{\m{}{l+1}}  \mel{\m{}{L+1}}{\H}{\Psi_T}
	\nn
	\approx& \frac{1}{\Z}
	\sum_{\{\m{}{l}\}_{2L+1}} \delta_{\Psi_T,\m{}{1}} \prod_{p=1}^{2L}  \mel**{\m{}{p}}{\oper{\rho}(p)}{\m{}{p+1}}  \mel{\m{}{2L+1}}{\H}{\Psi_T}
	\nn 
	%		=& \frac{1}{\Z} 
	%		\sum_{\{\m{}{p}\}_{2L-1}} \delta_{\Psi_T,\m{}{1}}
	%		\prod_{p=1}^{L-1} \mel**{\m{}{2p-1}}{e^{ -\tau \H_{odd}}}{\m{}{2p}}
	%		\mel**{\m{}{2p}}{e^{ -\tau \H_{even}}}{\m{}{2p+1}}
	%		\mel{\m{}{2L-1}}{ e^{ -\tau \H_{odd}} e^{ -\tau \H_{even}}\H}{\Psi_T}
	%		\nn 
	%		=& \frac{1}{\Z} 
	%		\sum_{\{\m{}{p}\}_{2L-1}}
	%		\delta_{\Psi_T,\m{}{1}}
	%		\prod_{p=1}^{2L-1}  \mel**{\m{}{p}}{\oper{\rho}(p)}{\m{}{p+1}}
	%		\mel{\m{}{2L-1}}{ e^{ -\tau \H_{odd}} e^{ -\tau \H_{even}}\H}{\Psi_T}
	%		\nn
	%		=& \frac{1}{\Z} 
	%		\sum_{\{\m{}{p}\}_{2L-1}}
	%		\delta_{\Psi_T,\m{}{1}}
	%		\prod_{p=1}^{2L-1}  \mel**{\m{}{p}}{\oper{\rho}(p)}{\m{}{p+1}}
	%		\mel{\m{}{2L-1}}{ e^{ -\tau \H_{odd}} 
		%			{\color{red} \sum_{\m{}{2L}}\ketbra{\m{}{2L}}}	
		%			e^{ -\tau \H_{even}}\H}{\Psi_T}
	%		\nn
	%		=& \frac{1}{\Z} 
	%		\sum_{\{\m{}{p}\}_{2L}}
	%		\delta_{\Psi_T,\m{}{1}}
	%		\prod_{p=1}^{2L-1}  \mel**{\m{}{p}}{\oper{\rho}(p)}{\m{}{p+1}}
	%		\mel{\m{}{2L-1}}{ e^{ -\tau \H_{odd}}}{\m{}{2L}}
	%		\mel{\m{}{2L}}{e^{ -\tau \H_{even}}\H}{\Psi_T}
	%		{\color{red}
		%			\frac{\mel{\m{}{2L}}{e^{-\tau \H_{even}}}{\Psi_T}}{ \mel{\m{}{2L}}{e^{-\tau \H_{even}}}{\Psi_T}}
		%		}
	%		\nn
	%		=& \frac{1}{\Z} 
	%		\sum_{\{\m{}{p}\}_{2L}}
	%		\delta_{\Psi_T,\m{}{1}}
	%		\prod_{p=1}^{2L-1}  \mel**{\m{}{p}}{\oper{\rho}(p)}{\m{}{p+1}}
	%		\mel{\m{}{2L-1}}{ e^{ -\tau \H_{odd}}}{\m{}{2L}}		
	%		{\color{red} \mel{\m{}{2L}}{e^{-\tau \H_{even}}}{\Psi_T}}
	%		\frac{\mel{\m{}{2L}}{e^{ -\tau \H_{even}}\H}{\Psi_T}}{ \color{red} \mel{\m{}{2L}}{e^{-\tau \H_{even}}}{\Psi_T}}
	%		\nn
	%		=& \frac{\sum_{\{\m{}{p}\}_{2L}}
		%			\delta_{\Psi_T,\m{}{1}}
		%			\prod_{p=1}^{2L}  \mel**{\m{}{p}}{\oper{\rho}(p)}{\m{}{p+1}}
		%			{\color{blue} \frac{\mel{\m{}{2L}}{e^{ -\tau \H_{even}}\H}{\Psi_T}}{\mel{\m{}{2L}}{e^{-\tau \H_{even}}}{\Psi_T}}}
		%		}{\sum_{\{\m{}{p}\}_{2L}}
		%			\delta_{\Psi_T,\m{}{1}}
		%			\prod_{p=1}^{2L}  \mel**{\m{}{p}}{\oper{\rho}(p)}{\m{}{p+1}}} 
\end{align}

\begin{align}
	\label{eq:expectation_energy_test2}
	\expval{\oper{E}}
	=& \frac{ \mel**{\Psi_T}{e^{-\frac{\beta}{2} \H} \H e^{- \frac{\beta}{2} \H }}{\Psi_T}}{\mel**{\Psi_T}{e^{-\beta \H}}{\Psi_T}} \nn 
	=& \frac{1}{\Z} \mel**{\Psi_T}{e^{-\beta \H} \H }{\Psi_T} \nn 
	=&\frac{1}{\Z}
	\sum_{\{\m{}{l}\}_{L}} \delta_{\Psi_T,\m{}{1}} \prod_{l=1}^{L-1}  \mel**{\m{}{l}}{ e^{ -\tau \H}}{\m{}{l+1}}  \mel{\m{}{L}}{ e^{ -\tau \H}\H}{\Psi_T} \nn
	\approx& \frac{1}{\Z}
	\sum_{\{\m{}{l}\}_{L}} \delta_{\Psi_T,\m{}{1}} \prod_{l=1}^{L-1}  \mel**{\m{}{l}}{ e^{ -\tau \H_{odd}} e^{ -\tau \H_{even}}}{\m{}{l+1}}  \mel{\m{}{L}}{ e^{ -\tau \H_{odd}} e^{ -\tau \H_{even}}\H}{\Psi_T}
	\nn 
	=& \frac{1}{\Z} 
	\sum_{\{\m{}{p}\}_{2L-1}} \delta_{\Psi_T,\m{}{1}}
	\prod_{p=1}^{L-1} \mel**{\m{}{2p-1}}{e^{ -\tau \H_{odd}}}{\m{}{2p}}
	\mel**{\m{}{2p}}{e^{ -\tau \H_{even}}}{\m{}{2p+1}}
	\mel{\m{}{2L-1}}{ e^{ -\tau \H_{odd}} e^{ -\tau \H_{even}}\H}{\Psi_T}
	\nn 
	=& \frac{1}{\Z} 
	\sum_{\{\m{}{p}\}_{2L-1}}
	\delta_{\Psi_T,\m{}{1}}
	\prod_{p=1}^{2L-1}  \mel**{\m{}{p}}{\oper{\rho}(p)}{\m{}{p+1}}
	\mel{\m{}{2L-1}}{ e^{ -\tau \H_{odd}} e^{ -\tau \H_{even}}\H}{\Psi_T}
	\nn
	=& \frac{1}{\Z} 
	\sum_{\{\m{}{p}\}_{2L-1}}
	\delta_{\Psi_T,\m{}{1}}
	\prod_{p=1}^{2L-1}  \mel**{\m{}{p}}{\oper{\rho}(p)}{\m{}{p+1}}
	\mel{\m{}{2L-1}}{ e^{ -\tau \H_{odd}} 
		{\color{red} \sum_{\m{}{2L}}\ketbra{\m{}{2L}}}	
		e^{ -\tau \H_{even}}\H}{\Psi_T}
	\nn
	=& \frac{1}{\Z} 
	\sum_{\{\m{}{p}\}_{2L}}
	\delta_{\Psi_T,\m{}{1}}
	\prod_{p=1}^{2L-1}  \mel**{\m{}{p}}{\oper{\rho}(p)}{\m{}{p+1}}
	\mel{\m{}{2L-1}}{ e^{ -\tau \H_{odd}}}{\m{}{2L}}
	\mel{\m{}{2L}}{e^{ -\tau \H_{even}}\H}{\Psi_T}
	{\color{red}
		\frac{\mel{\m{}{2L}}{e^{-\tau \H_{even}}}{\Psi_T}}{ \mel{\m{}{2L}}{e^{-\tau \H_{even}}}{\Psi_T}}
	}
	\nn
	=& \frac{1}{\Z} 
	\sum_{\{\m{}{p}\}_{2L}}
	\delta_{\Psi_T,\m{}{1}}
	\prod_{p=1}^{2L-1}  \mel**{\m{}{p}}{\oper{\rho}(p)}{\m{}{p+1}}
	\mel{\m{}{2L-1}}{ e^{ -\tau \H_{odd}}}{\m{}{2L}}		
	{\color{red} \mel{\m{}{2L}}{e^{-\tau \H_{even}}}{\Psi_T}}
	\frac{\mel{\m{}{2L}}{e^{ -\tau \H_{even}}\H}{\Psi_T}}{ \color{red} \mel{\m{}{2L}}{e^{-\tau \H_{even}}}{\Psi_T}}
	\nn
	=& \frac{\sum_{\{\m{}{p}\}_{2L}}
		\delta_{\Psi_T,\m{}{1}}
		\prod_{p=1}^{2L}  \mel**{\m{}{p}}{\oper{\rho}(p)}{\m{}{p+1}}
		{\color{blue} \frac{\mel{\m{}{2L}}{e^{ -\tau \H_{even}}\H}{\Psi_T}}{\mel{\m{}{2L}}{e^{-\tau \H_{even}}}{\Psi_T}}}
	}{\sum_{\{\m{}{p}\}_{2L}}
		\delta_{\Psi_T,\m{}{1}}
		\prod_{p=1}^{2L}  \mel**{\m{}{p}}{\oper{\rho}(p)}{\m{}{p+1}}} 
\end{align}

\begin{align}
	\frac{\mel{\m{}{P}}{e^{-\tau \H_{even}}\H}{\Psi_T}}{\mel{\m{}{P}}{e^{-\tau \H_{even}}}{\Psi_T}}
	=&\frac{\mel{\m{}{P}}{e^{-\tau \H_{even}}~(\H_{odd}+\H_{even})}{\va{0}}}{\mel{\m{}{P}}{e^{-\tau \H_{even}}}{\Psi_T}} \nn
	=&\frac{\mel{\m{}{P}}{e^{-\tau \H_{even}}\H_{odd}+e^{-\tau \H_{even}}\H_{even}}{\Psi_T}}{\mel{\m{}{P}}{e^{-\tau \H_{even}}}{\Psi_T}} \nn
	=&\frac{\mel{\m{}{P}}{e^{-\tau \H_{even}}\H_{odd}}{\Psi_T}+\mel{\m{}{P}}{e^{-\tau \H_{even}}\H_{even}}{\Psi_T}}{\mel{\m{}{P}}{e^{-\tau \H_{even}}}{\Psi_T}}
\end{align}

\begin{align}
	\frac{\mel{\m{}{P}}{e^{-\tau \H_{even}}\H_{even}}{\Psi_T}}{\mel{\m{}{P}}{e^{-\tau \H_{even}}}{\Psi_T}}
	=&\frac{\mel{\m{}{P}}{e^{-\tau \H_{23}}e^{-\tau \H_{45}}e^{-\tau \H_{67}}\dots (\H_{23}+\H_{45}+\H_{67}+\dots)}{\Psi_T}}{\mel{\m{}{P}}{e^{-\tau \H_{even}}}{\Psi_T}} \nn
	=& \sum_{n \text{ even}}^{N-1}
	\frac{\mel{\m{n}{P},\m{n+1}{P}}{e^{-\tau \H_{n,n+1}}\H_{n,n+1}}{\psi_{t_n},\psi_{t_{n+1}}}
	}{
		\mel{\m{n}{P},\m{n+1}{P}}{e^{-\tau \H_{even}}}{\psi_{t_n},\psi_{t_{n+1}}}
	}
\end{align}

\begin{align}
	\frac{\mel{\m{}{P}}{e^{-\tau \H_{even}}\H_{odd}}{\Psi_T}}{\mel{\m{}{P}}{e^{-\tau \H_{even}}}{\Psi_T}}
	=&
	\frac{\mel{\m{}{P}}{e^{-\tau \H_{even}} 
			{\color{red} \sum_{\m{}{\alpha}} \ketbra{\m{}{\alpha}}}		
			\H_{odd}}{\Psi_T}}{\mel{\m{}{P}}{e^{-\tau \H_{even}}}{\Psi_T}}
	\nn	
	=&
	\sum_{\m{}{\alpha}} \frac{\mel{\m{}{P}}{e^{-\tau \H_{even}}}{\m{}{\alpha}}		
	}{\mel{\m{}{P}}{e^{-\tau \H_{even}}}{\Psi_T}} \sum_{n \text{ odd}}^{N-1} \mel{\m{n}{\alpha},\m{n+1}{\alpha}}{\H_{odd_{n,n+1}}}{\psi_{t_n},\psi_{t_{n+1}}}
	\nn	
	=&
	\sum_{\m{}{\alpha}}
	\prod_{i \text{ even}}
	\frac{\mel{\m{i}{P},\m{i+1}{P}}{e^{-\tau \H_{i,i+1}}}{\m{i}{\alpha},\m{i+1}{\alpha}}		
	}{\mel{\m{i}{P},\m{i+1}{P}}{e^{-\tau \H_{i,i+1}}}{\psi_{t_i},\psi_{t_{i+1}}}}
	\sum_{n \text{ odd}} \mel{\m{n}{\alpha},\m{n+1}{\alpha}}{\H_{n,n+1}}{\psi_{t_n},\psi_{t_{n+1}}}
	\nn
	=&
	\sum_{\m{1}{\alpha}}\sum_{\m{2}{\alpha}}\sum_{\m{3}{\alpha}} \dots \sum_{\m{N}{\alpha}}
	\prod_{i \text{ even}}
	\rho^{\alpha}_{i,i+1}		
	\sum_{n \text{ odd}} \gamma^{\alpha}_{n,n+1}
	\nn
	=&		
	\sum_{n \text{ odd}} ~~   
	\sum_{\m{1}{\alpha}}\sum_{\m{2}{\alpha}}\sum_{\m{3}{\alpha}} \dots \sum_{\m{N}{\alpha}}
	\prod_{i \text{ even}}
	\rho^{\alpha}_{i,i+1} \gamma^{\alpha}_{n,n+1}
	%	=&
	%	\sum_{n \text{ odd}}
	%	\sum_{\m{}{\alpha}}
	%	\prod_{i \text{ even}}
	%	\frac{\mel{\m{i}{P},\m{i+1}{P}}{e^{-\tau \H_{i,i+1}}}{\m{i}{\alpha},\m{i+1}{\alpha}}		
		%	}{\mel{\m{i}{P},\m{i+1}{P}}{e^{-\tau \H_{i,i+1}}}{\psi_{t_i},\psi_{t_{i+1}}}} \mel{\m{n}{\alpha},\m{n+1}{\alpha}}{\H_{n,n+1}}{\psi_{t_n},\psi_{t_{n+1}}}
	%%%
	%	=&\frac{\mel{\m{}{P}}{e^{-\tau \H_{23}}e^{-\tau \H_{45}}e^{-\tau \H_{67}}\dots (\H_{12}+\H_{34}+\H_{56}+\dots)}{\Psi_T}}{\mel{\m{}{P}}{e^{-\tau \H_{even}}}{\Psi_T}} \nn
	%	=&\frac{\mel{\m{}{P}}{\H_{12}e^{-\tau \H_{23}}e^{-\tau \H_{45}}e^{-\tau \H_{67}}\dots + e^{-\tau \H_{23}}\H_{34}e^{-\tau \H_{45}}e^{-\tau \H_{67}}\dots + e^{-\tau \H_{23}}e^{-\tau \V{45}}\V{56}e^{-\tau \V{67}}\dots}{\Psi_T}}{\mel{\m{}{P}}{e^{-\tau \H_{even}}}{\Psi_T}} \nn
	%	=&
	%	\frac{\mel{\m{1}{P},\m{2}{P}, \m{3}{P}}{ \V{12}e^{-\tau \V{23}}}{0,0,0}
		%	}{
		%		\mel{\m{2}{P}, \m{3}{P}}{e^{-\tau \V{23}}}{0,0}
		%	}+\nn
	%	&+
	%	\sum_{n \text{ even}}^{N-3}
	%	\frac{\mel{\m{n}{P},\m{n+1}{P}, \m{n+2}{P},\m{n+3}{P}}{e^{-\tau \V{n,n+1}}\,\V{n+1,n+2}\,e^{-\tau \V{n+2,n+3}}}{0,0,0,0}
		%	}{
		%		\mel{\m{n}{P},\m{n+1}{P}}{e^{-\tau \V{n,n+1}}}{0,0}\mel{\m{n+2}{P},\m{n+3}{P}}{e^{-\tau \V{n+2,n+3}}}{0,0}
		%	}+\nn&+
	%	\frac{\mel{\m{N-2}{P},\m{N-1}{P}, \m{N}{P}}{ e^{-\tau \V{N-2,N-1}}\V{N-1,N}}{0,0,0}
		%	}{
		%		\mel{\m{N-2}{P}, \m{N-1}{P}}{e^{-\tau \V{N-2,N-1}}}{0,0}
		%	}
\end{align}







\newpage

\newpage

\subsubsection{System of $N=2$ particles}

Here, the simplest case would be a system of $N=2$ planar rotors. Then, eq. \eqref{eq:partition_function_PIMC_Trotter_2} would reduce to
\begin{align}
	\label{eq:partition_function_PIMC_N=2_ex}
	\Z =& \sum_{\{\m{1}{l}, \m{2}{l}\}_{L}} \prod_{l=1}^{L} e^{-\frac{\tau}{2}(\m{1}{l})^2-\frac{\tau}{2}(\m{2}{l})^2} \mel**{\m{1}{l},\m{2}{l}}{ e^{ -\tau \V{12}}}{\m{1}{l+1},\m{2}{l+1}} e^{-\frac{\tau}{2}(\m{1}{l+1})^2-\frac{\tau}{2}(\m{2}{l+1})^2} \, ,
\end{align}
graphically depicted as in Fig. \ref{fig:PIMC_grid_N=2}.
\begin{figure}[ht]
	\centering
	\incfig{PIMC_grid_N=2}
	\caption{Scheme showing the graphical representation of the grid defined on Eq. \eqref{eq:partition_function_PIMC_N=2_ex} for a system of $N=2$ planar rotors, where the dots represents the particles, and the hollow rectangles, the two particle operators.}
	\label{fig:PIMC_grid_N=2}
\end{figure}



\subsection{Idk how to name this section}
\label{sec:}

From Eq. \eqref{eq:part_function_expanded}, or even Eq. \eqref{eq:partition_function_PIMC_N=2_ex} we see that all the interactions were reduced to two particle interactions and all the partition function is represented by a product of the matrix elements of two particle operators representing the interaction. Here, in order to simplify the notation we can treat pair of particles as a new object that will be indexed as $M^{p} = (\m{1}{p}+\mmax)\cdot (2\mmax+1) + (\m{2}{p}+\mmax)$

\alert{... continue this part to check the Gibbs sampling process...}



\newpage


\begin{align}
	\label{eq:expectation_energy_test}
	\expval{\oper{E}}
	=& \frac{ \mel**{\Psi_T}{e^{-\frac{\beta}{2} \H} \H e^{- \frac{\beta}{2} \H }}{\Psi_T}}{\mel**{\Psi_T}{e^{-\beta \H}}{\Psi_T}} \nn 
	=& \frac{1}{\Z} \mel**{\Psi_T}{e^{-\beta \H} \H }{\Psi_T}\nn 
	=& \frac{1}{\Z} \sum_{\{\m{}{l}\}_{L}} \delta_{\Psi_T,\m{}{1}} \prod_{l=1}^{L-1} \mel**{\m{}{l}}{e^{-\tau \H}}{\m{}{l+1}} \mel{\m{}{L}}{e^{-\tau \H}\H}{\Psi_T}\nn 
	=& \frac{
		\sum_{\{\m{}{p}\}_{P}} \delta_{\Psi_T,\m{}{1}} \tilde{G}_{\tau}(\m{}{1}, \m{}{P-1}) \bra{\m{}{P-1}} e^{-\tau \V{odd}} \ketbra{\m{}{P}} e^{-\tau \V{even}}\H\ket{\Psi_T}
	}{
		\sum_{\{\m{}{p}\}_{P+1}} \delta_{\Psi_T,\m{}{1}} \tilde{G}_{\tau}(\m{}{1}, \m{}{P+1})\delta_{\m{}{P+1},\Psi_T}}  \nn 
	=& \frac{
		\sum_{\{\m{}{p}\}_{P}} \delta_{\Psi_T,\m{}{1}} \tilde{G}_{\tau}(\m{}{1}, \m{}{P-1}) \bra{\m{}{P-1}} e^{-\tau \V{odd}} \ketbra{\m{}{P}} e^{-\tau \V{even}}\H\ket{\Psi_T}
		\times 
		{\color{red}
			\frac{
				\sum_{\m{}{P+1}} \mel{\m{}{P}}{e^{-\tau \V{even}}}{\m{}{P+1}}\delta_{\m{}{P+1},\Psi_T}
			}{
				\sum_{\m{}{P+1}} \mel{\m{}{P}}{e^{-\tau \V{even}}}{\m{}{P+1}}\delta_{\m{}{P+1},\Psi_T}
			}
		}
	}{
		\sum_{\{\m{}{p}\}_{P+1}} \delta_{\Psi_T,\m{}{1}} \tilde{G}_{\tau}(\m{}{1}, \m{}{P+1})\delta_{\m{}{P+1},\Psi_T}}\nn
	&\hspace{1cm}
	\nn
	=& \frac{
		\sum_{\{\m{}{p}\}_{P+1}} \delta_{\Psi_T,\m{}{1}} \tilde{G}_{\tau}(\m{}{1}, \m{}{P+1})\delta_{\m{}{P+1},\Psi_T}
		\times 
		\frac{\mel{\m{}{P}}{e^{-\tau \V{even}}\H}{\Psi_T}}{\mel{\m{}{P}}{e^{-\tau \V{even}}}{\Psi_T}}
	}{
		\sum_{\{\m{}{p}\}_{P+1}} \delta_{\Psi_T,\m{}{1}} \tilde{G}_{\tau}(\m{}{1}, \m{}{P+1})\delta_{\m{}{P+1},\Psi_T}
	}
	\, ,
\end{align}
Then
\begin{align}
	\frac{\mel{\m{}{P}}{e^{-\tau \V{even}}\H}{\Psi_T}}{\mel{\m{}{P}}{e^{-\tau \V{even}}}{\Psi_T}}
	=&\frac{\mel{\m{}{P}}{e^{-\tau \V{even}} (\K{}+ \V{})}{\Psi_T}}{\mel{\m{}{P}}{e^{-\tau \V{even}}}{\Psi_T}} \nn
\end{align}







\newpage	
	
	%-------------------------------------------------------------------
	%%%%%%%%%% REFERENCES %%%%%%%%%%%%%%%%%%%%%%
	\bibliography{references.bib}
	\bibliographystyle{ieeetr}
	
	\newpage
	\appendix
	
	\section{Commutators between Hamiltonian terms}
	\label{sec:commutators_calculations}
	
	From eqs. \eqref{eq:L_action}, \eqref{eq:E_pm_action}, \eqref{eq:L,E_pm_comm} and \eqref{eq:Ep,Em_comm} we have
	
	\begin{align}
		\comm{\L_i^2}{\E{\pm}{i}}
		=&\L_i^2\E{\pm}{i} - \E{\pm}{i}\L_i^2 \nn
		=&\L_i \L_i \E{\pm}{i} {\color{red} - \L_i \E{\pm}{i} \L_i + \L_i \E{\pm}{i} \L_i} - \E{\pm}{i} \L_i \L_i \nn
		=&\L_i \comm{\L_i}{\E{\pm}{i}} + \comm{\L_i}{\E{\pm}{i}} \L_i \nn
		=& \pm \left( \L_i \E{\pm}{i} + \E{\pm}{i}\L_i \right) \nn
		=& \pm \left( 2 \L_i \E{\pm}{i} - \comm{\L_i}{\E{\pm}{i}} \right) \nn
		=& \left(\pm 2 \L_i - 1\right)\E{\pm}{i} \nn
		\neq & 0 \,.
	\end{align}           
	               
	\begin{align}
		\comm{\L_i^2}{\E{\pm}{i}\E{\pm}{j}}
		=& \comm{\L_i^2}{\E{\pm}{i}}\E{\pm}{j} \nn
		=& \left(\pm 2 \L_i - 1\right)\E{\pm}{i}\E{\pm}{j} \nn
		\neq & 0 \,.
	\end{align}
	
	\begin{align}
		\comm{\L_i^2}{\E{\mp}{i}\E{\pm}{j}}
		=& \comm{\L_i^2}{\E{\mp}{i}}\E{\pm}{j} \nn
		=& \left(\mp 2 \L_i - 1\right)\E{\mp}{i}\E{\pm}{j} \nn
		\neq & 0 \,.
	\end{align}
	
	\label{sec:comm_V_terms}
	Using Eq. \eqref{eq:dipole-dipole_interaction},
	\begin{align}
		\comm{\K{i}}{\V{ij}} =& \frac{g}{4}\comm{\L_i^2}{3\E{+}{i}\E{+}{j}+\E{+}{i}\E{-}{j}+\E{-}{i}\E{+}{j}+3\E{-}{i}\E{-}{j}} \nn
		=& \frac{g}{4}
		\left\{
		3\comm{\L_i^2}{\E{+}{i}\E{+}{j}}+\comm{\L_i^2}{\E{+}{i}\E{-}{j}}+\comm{\L_i^2}{\E{-}{i}\E{+}{j}}+3\comm{\L_i^2}{\E{-}{i}\E{-}{j}}
		\right\} \nn
		=&\frac{g}{4}
		\left\{
		3\left(2 \L_i - 1\right) \E{+}{i}\E{+}{j} + \left(2 \L_i - 1\right) \E{+}{i}\E{-}{j}+ \left(-2 \L_i - 1\right)\E{-}{i}\E{+}{j} +3\left(- 2 \L_i - 1\right) \E{-}{i}\E{-}{j}
		\right\} \nn
		=& 2 \L_i \left\{
		\frac{2g}{4} \left[3\E{+}{i}\E{+}{j}+\E{+}{i}\E{-}{j}\right] - \V{ij} \right\}- \V{ij}
		\nn
		=& g \L_i \left[3\E{+}{i}\E{+}{j}+\E{+}{i}\E{-}{j}\right] - \left(2 \L_i + 1\right) \V{ij}
		\nn
		\neq& 0 \,.
	\end{align}
	Also, the commutator $\comm{\V{ij}}{\V{jl}}$ follows
	\begin{align}
		\comm{\V{ij}}{\V{jl}} =& \frac{g^2}{16}\comm{3\E{+}{i}\E{+}{j}+\E{+}{i}\E{-}{j}+\E{-}{i}\E{+}{j}+3\E{-}{i}\E{-}{j}}{3\E{+}{j}\E{+}{l}+\E{+}{j}\E{-}{l}+\E{-}{j}\E{+}{l}+3\E{-}{j}\E{-}{l}} \nn
		=&\frac{g^2}{16}\comm{\E{+}{i}(3\E{+}{j}+\E{-}{j})+\E{-}{i}(\E{+}{j}+3\E{-}{j})}{(3\E{+}{j}+\E{-}{j})\E{+}{l}+(\E{+}{j}+3\E{-}{j})\E{-}{l}} \nn
		=&\frac{g^2}{16}\E{+}{i}\comm{3\E{+}{j}+\E{-}{j}}{\E{+}{j}+3\E{-}{j}}\E{-}{l} + \frac{g^2}{16}\E{-}{i}\comm{\E{+}{j}+3\E{-}{j}}{3\E{+}{j}+\E{-}{j}}\E{+}{l}
		\nn
		=&\frac{g^2}{16}\E{+}{i}\left( 9 \comm{\E{+}{j}}{\E{-}{j}} + \comm{\E{-}{j}}{\E{+}{j}}\right)\E{-}{l} + 
		\frac{g^2}{16}\E{-}{i}\left( \comm{\E{+}{j}}{\E{-}{j}} + 9 \comm{\E{-}{j}}{\E{+}{j}}\right)\E{+}{l}
		\nn
		=&\frac{g^2}{2}\E{+}{i}\comm{\E{+}{j}}{\E{-}{j}}\E{-}{l} - 
		\frac{g^2}{2}\E{-}{i}\comm{\E{+}{j}}{\E{-}{j}}\E{+}{l}
		\nn
		=&\frac{g^2}{2}\underbrace{\comm{\E{+}{j}}{\E{-}{j}}}_{\color{Red} =0 \text{ from Eq. \eqref{eq:Ep,Em_comm}}} \left(\E{+}{i}\E{-}{l} -\E{-}{i}\E{+}{l} \right)
		\nn
		=&0 \, .
	\end{align}


\end{document}
